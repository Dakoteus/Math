%!TeX program=pdflatex
\documentclass[titlepage]{article}
 
\usepackage[margin=1in]{geometry} 
\usepackage{amsmath,amsthm,amssymb,fancyhdr,gensymb,arydshln}
\pagestyle{fancy}

\newenvironment{theorem}[2][Theorem]{\begin{trivlist}
\item[\hskip \labelsep {\bfseries #1}\hskip \labelsep {\bfseries #2.}]}{\end{trivlist}}
\newenvironment{lemma}[2][Lemma]{\begin{trivlist}
\item[\hskip \labelsep {\bfseries #1}\hskip \labelsep {\bfseries #2.}]}{\end{trivlist}}
\newenvironment{exercise}[2][Exercise]{\begin{trivlist}
\item[\hskip \labelsep {\bfseries #1}\hskip \labelsep {\bfseries #2.}]}{\end{trivlist}}
\newenvironment{problem}[2][Problem]{\begin{trivlist}
\item[\hskip \labelsep {\bfseries #1}\hskip \labelsep {\bfseries #2.}]}{\end{trivlist}}
\newenvironment{question}[2][Question]{\begin{trivlist}
\item[\hskip \labelsep {\bfseries #1}\hskip \labelsep {\bfseries #2.}]}{\end{trivlist}}
\newenvironment{corollary}[2][Corollary]{\begin{trivlist}
\item[\hskip \labelsep {\bfseries #1}\hskip \labelsep {\bfseries #2.}]}{\end{trivlist}}


\begin{document}
 
% --------------------------------------------------------------
%                         Start here
% --------------------------------------------------------------
 
%\title{Weekly Homework II}%replace X with the appropriate number
%\author{Dakota Wicker\\ %replace with your name
%Abstract Algebra I} %if necessary, replace with your course title
%\maketitle
%\clearpage
\fancyhf{}
\fancyhead[RO,RE]{Abstract I}
\fancyhead[LO,LE]{Dakota Wicker}
\fancyhead[CO,CE]{Homework IV}
\cfoot{\thepage}

\begin{problem}{1}
	Use mathematical induction to prove that
	$$ 3 + 3\cdot4 + 3\cdot4^2 + ... + 3\cdot4^n = 4^{n+1} - 1$$
	for all integers $n \geq 0$. 
	\\ \\
	\begin{proof}
		Using the steps of induction I show that this holds for the base case $n=0$
		$$4^{0+1}-1 = 4-3 = 3$$
	Now, I assume this holds for $n=k$, $k \geq 0$, that is 
	$$3 + 3\cdot4 + 3\cdot4^2 + ... + 3\cdot4^n = 4^{n+1} - 1 $$
	 then I want to show this also holds for $n=k+1$.
	 Factoring the L.H.S of this equation
$$3+3\cdot4+3\cdot4^2+...+3\cdot4^n+3\cdot4^{n+1} = 4^{n+2}-1$$
I get
$$3+4(3+4\cdot3+...+4^{n-1}\cdot3+4^{n}\cdot3) = 4^{n+2}-1$$
Doing algebraic manipulation I get
$$4+4(3+4\cdot3+...+4^{n-1}\cdot3+4^{n}\cdot3) = 4^{n+2}$$
it follows that
$$1+1(3+4\cdot3+...+4^{n-1}\cdot3+4^{n}\cdot3) = 4^{n+1}$$
and 
$$3+3\cdot4+...+3\cdot4^{n-1} + 3\cdot4^n = 4^{n+1} - 1$$
This shows that the identity still holds when $n=k+1$, and the induction is completed.
\end{proof}
\end{problem}

\begin{problem}{2}
	Solve the congruence
	$$175x \equiv 234 \ \text{mod(603)}$$
	\\ \\
	To find $x$, I can rewrite the problem as 
	\begin{equation}175^{-1} 175x \equiv 175^{-1} 234 \ \text{(mod 603)} \end{equation}
	and to find $175^{-1}$, I can use the extended euclidean algorithm to find $s$ and $t$ such that
	$$175s + 603t \equiv 1 \ \text{(mod 603)} $$

$$	\begin{array}{r|r|r|r|r}
	s_k&t_k&q_k& & \\\hline
	0&1& & &  \\\hline
	1 & 0 & 3& 603 & 175\\\hline
	-3 & 1 & 2 & 525 & 156 \\\hline
	7 & -2 & 4 & 78 & 19 \\\hline
	-31 & 9 & 9 & 76 & 18 \\\hline
	286 & -83 & 2 & 2 & 1 \\\hline
	&&&2&\\\hline
	&&&0&
	\end{array}$$
	So, plugging $s$ and $t$ into the previous equation I get,
	$$175(286) + 603(-83) \equiv 1 \ \text{(mod 603)}$$
	which means that 
	$$175^{-1} \equiv 286 \text{(mod 603)}$$
	Finally, substituting the inverse into (1) I get
	$$x \equiv 286\cdot 234 \equiv 594 \ \text{(mod 603)}$$
\end{problem}

\begin{problem}{3}
	 Approximate, in 4 decimal places, all the 6th roots of $-3+8i$. \\ \\
	 Using De Moivre's theorem, I will use the fact
	 $$z^{\frac{1}{n}} = r^{\frac{1}{n}}\cdot \text{cis}{\left(\frac{\theta}{n} + \frac{2k\pi}{n}\right)}$$
	 to find the 6th roots of $-3+8i$ by substituting $r,n,$ and $k$ with the corresponding values:
	 $$z=-3+8i$$
	 $$\theta = \pi - \tan^{-1}{(\frac{8}{3})} $$
	 $$r = \sqrt{(-3)^2 + (-8)^2} = \sqrt{73}$$
	 $$n = 6$$
	 $$k=0,1,2,...,n-1$$
	 Substituting these values, I get the roots to be
	\begin{align*}
		z^{\frac{1}{6}} &= \sqrt[12]{73} \cdot \text{cis}\left(\frac{\pi - \tan^{-1}\left(\frac{8}{3}\right)}{6} + \frac{2\cdot0\cdot\pi}{6} \right) \approx \phantom{-}1.3565 + 0.4519i\\
	 z^{\frac{1}{6}} &= \sqrt[12]{73} \cdot \text{cis}\left(\frac{\pi - \tan^{-1}\left(\frac{8}{3}\right)}{6} + \frac{2\cdot1\cdot\pi}{6} \right) \approx \phantom{-}0.2869 + 1.4007i
\\
	 z^{\frac{1}{6}} &= \sqrt[12]{73} \cdot \text{cis}\left(\frac{\pi - \tan^{-1}\left(\frac{8}{3}\right)}{6} + \frac{2\cdot2\cdot\pi}{6} \right) \approx -1.0696 + 0.9488i
	 \\
	 z^{\frac{1}{6}} &= \sqrt[12]{73} \cdot \text{cis}\left(\frac{\pi - \tan^{-1}\left(\frac{8}{3}\right)}{6} + \frac{2\cdot3\cdot\pi}{6} \right) \approx -1.3565 - 0.4519i
\\
	 z^{\frac{1}{6}} &= \sqrt[12]{73} \cdot \text{cis}\left(\frac{\pi - \tan^{-1}\left(\frac{8}{3}\right)}{6} + \frac{2\cdot4\cdot\pi}{6} \right) \approx -0.2869 - 1.4007i
\\
z^{\frac{1}{6}} &= \sqrt[12]{73} \cdot \text{cis}\left(\frac{\pi - \tan^{-1}\left(\frac{8}{3}\right)}{6} + \frac{2\cdot5\cdot\pi}{6} \right) \approx \phantom{-}1.0696 - 0.9488i
\end{align*}
\end{problem}

\begin{problem}{4}
	What are the primitive 15th roots of unity? Leave your answers in the form of $\omega^k$ for some appropriate complex number $\omega$. Be sure to describe what $\omega$ represents.
	\\
	\\
	The primitive 15th roots of unity are the roots of unity where $n = 15$,  $\omega$ is a number such that $\omega^{n}= 1$ and $k=1,2,...,n-1$.
	When GCD($n$,$k$) = 1, in other words, when $k$ and $n$ are relatively prime, $\omega^{k}$ is a root of unity. So, the 15th roots of unity are
	$$\omega^2, \omega^4, \omega^7, \omega^8, \omega^{11}, \omega^{13}, \omega^{14}$$
\end{problem}

\begin{problem}{5}
How many fourth roots of the matrix
$$D = \begin{bmatrix}-\frac{1}{2} & -\frac{\sqrt{3}}{2} \\ \phantom{-} \frac{\sqrt{3}}{2} & -\frac{1}{2} \end{bmatrix} $$
are there? What are they? Leave the answers in the exact form.
\\ \\
There are 4 fourth roots to this matrix. Using De Moivre's theorem I can use the fact that 
 \begin{equation}z^{\frac{1}{n}} = r^{\frac{1}{n}}\cdot \text{cis}{\left(\frac{\theta}{n} + \frac{2k\pi}{n}\right)} \end{equation} to find the 4th roots of the complex number $z = -\frac{1}{2} + \frac{\sqrt{3}}{2}$ which is isomorphic to the matrix D. I can find the roots by substituting $r, n$ and $k$ into (1) where 
 $$\theta = \pi - \tan^{-1}(\frac{\frac{\sqrt{3}}{2}}{\frac{1}{2}}) = \pi-\tan^{-1}{\left(\sqrt{3}\right)} = \pi - \frac{\pi}{3} = \frac{2\pi}{3} $$
 $$r = \sqrt{\left(-\frac{1}{2}\right)^2 + \left(\frac{\sqrt{3}}{2}\right)^2} = \sqrt{\frac{1}{4} + \frac{3}{4}} = \sqrt{1} = 1$$
 $$n = 4$$
 $$ k = 0,1,2,3$$
 Substituting these values I get the roots to be
$$z^{\frac{1}{4}} = 1\cdot \text{cis}\left(\frac{2\pi}{12} + \frac{2\cdot0\cdot\pi}{4}\right) $$
$$z^{\frac{1}{4}} = 1\cdot \text{cis}\left(\frac{2\pi}{12} + \frac{2\cdot1\cdot\pi}{4}\right) $$
$$z^{\frac{1}{4}} = 1\cdot \text{cis}\left(\frac{2\pi}{12} + \frac{2\cdot2\cdot\pi}{4}\right) $$
$$z^{\frac{1}{4}} = 1\cdot \text{cis}\left(\frac{2\pi}{12} + \frac{2\cdot3\cdot\pi}{4}\right) $$
\end{problem}

\begin{problem}{6}
Define $\odot$ on $\mathbb{C}^*$ according to
$$(a+bi) \odot (c+di) = ac + bdi.$$
\begin{itemize}
\item[(a)] Is $\odot$ well-defined? In other words, is $\mathbb{C}^*$ closed under $\odot$?
\item[(b)] Is $\odot$ associative? If you think it is, prove it. If you do not thing so, explain, or provide a counterexample.
\item[(c)] Find, if possible, an element $e$ such that $e\odot z =z \ \text{for any} \ z \in \mathbb{C}^*.$ Or, explain why such an element does not exist. 
\item[(d)] Based on the identity element you found in (c), find the inverse of a typical element $z$  in $\mathbb{C}^*$. Does the inverse always exist?
\item[(e)] Is $\langle \mathbb{C}^*,\odot\rangle$ a group? Explain.
\end{itemize}
\hrulefill 
\begin{itemize}
	
	\item[(a)] $\mathbb{C}^*$ is not closed under $\odot$ because there are two elements in $\mathbb{C}^*$ I can use with the $\odot$ operator to get an element which is not in $\mathbb{C}^*$. For example, $0+1i\odot 1+0i = 0+0i$ and $0+0i \notin \mathbb{C}^*$. Therefore $\odot$ is not well-defined
	\item[(b)] $\odot$ is associative. To show that $\odot$ is associative, I let $A=a+bi, B=c+di, C=e+fi $ where $A,B,C \in \mathbb{C}^*$ and will show that 
		$$(A\odot B)\odot C = A\odot(B\odot C)$$
		Since,
		$$(A\odot B)\odot C = ac + bdi \odot C = ace + bdfi$$
		and
		$$ A\odot(B\odot C) = A \odot (ce + dfi) = cea + dfbi$$
		because real numbers under multiplication are known to be commutative it follows that $(A\odot B)\odot C = A\odot(B\odot C)$. Therefore $\odot$ is associative.
	\item[(c)] The element $e\in\mathbb{C}^*$ such that $e \odot z = z$ exists. This can be shown by rewriting $e\odot z =z $ as
		$$e_1 + e_2i \odot z_1 + z_2i = e_1z_1 + e_2z_2 = z_1 + z_2i$$
		it is clear that $e_1+e_2i$ must equal $1+1i$ which is in $\mathbb{C}^*$. Therefore $e = 1+1i$.
	\item[(d)] The inverse is the element $A^{-1}$ such that $A \odot A^{-1} = e$. In this case, an inverse does not always exist. I can show this by rewriting $A \odot A^{-1} = e$ to be 
		$$a+bi \odot a'+b'i = aa' + bb'i = 1 + 1i = e$$ It would follow that $a'$ would be the multiplicative inverse of $a$ and $b'$ would be the multiplicative inverse of $b$ leaving the inverse to be $A^{-1} = \frac{1}{a} + \frac{1}{b}i$. But this is not always the case because either $a$ or $b$ could be equal to zero which there is no multiplicative inverse of. Therefore the inverse does not always exist.
	\item[(e)]  $\langle \mathbb{C}^*, \odot\rangle$ is not a group because it is not closed under $\odot$ and every element in $\langle \mathbb{C}^*,\odot\rangle$ does not necessarily have an inverse.
\end{itemize}
\end{problem}

\begin{problem}{7}
	Let $S = \mathbb{R} - \{-1\}.$ In other words, $S$ is the set of all real numbers except -1. Define a binary operation $*$ on $S$ by
	$$ a*b = a+b+ab.$$
	Show that $\langle S, * \rangle$ is a group, as follows:
	\begin{itemize}
		\item[(a)] Establish closure using a proof by contradiction.
		\item[(b)] Show that $*$ is associative
		\item[(c)] Find the identity element
		\item[(d)] Find the inverse of $a$. Be sure to show that it is an element of $S$.
		\item[(e)] What is your conclusion about $\langle S,*\rangle$?
	\end{itemize}
	\hrulefill
	\begin{itemize}
		\item[(a)] Suppose that $S$ was not closed under $*$. Then there is an $a*b = -1$ because $a*b = a+b+ab$ is closed under $\mathbb{R}$ because multiplication and addition is closed. So the only element that would make $S$ not closed is $-1$ because $-1 \in \mathbb{R}$ and $-1 \notin \mathbb{R} - \{-1\}$. Furthermore, if 
			$$a*b = a + b + ab = -1$$
			Then when I solve for $a$ I get,
		
			$$	a+ab = -1-b $$ $$ a(1+b) = -1-b $$ $$ a = \frac{-1-b}{\phantom{-}1+b} =\frac{-(1+b)}{\phantom{-}(1+b)} =-1 $$
			Since $a$ must be equal to $-1$, and $-1 \notin \mathbb{R}-\{-1\}$, this forms a contradiction with the assumption that $a\in \mathbb{R}-\{-1\}$. Therefore $S$ must be closed.
		
		\item[(b)] To show that $*$ is associative, I will show that $(a*b)*c = a*(b*c)$, where $a,b,c \in S$
			$$(a*b)*c = (a+b+ab)*c = (a+b+ab) + c + (a+b+ab)c = a+b+c+ab+ac+bc+abc$$
	and
	$$a*(b*c) = a*(b + c + bc) = a + (b+c+bc) + a(b+c+bc) =(a+b+c+bc) + (ab+ac+abc) = a+b+c+ab+ac+bc+abc$$
	This shows that $(a*b)*c = a*(b*c)$.
	\item[(c)] The identity element $e \in S$ is the element such that $e*a = a$. That is,
		$$e*a = e + a + ea$$
		If I let $e=0$ it becomes clear that 
		$$0*a = 0 + a + 0a = a$$
		this shows that $e=0$ is the identity.
	\item[(d)] The inverse of an element $a$, denoted $a^{-1}$ is an element in $S$ such that $a*a^{-1} = e$. To find the inverse, I rewrite $a*a^{-1}=e$ as 
		$$a*a^{-1}=a + a^{-1} + aa^{-1} = 0$$
		Now, subtracting $a$ on both sides I get
		$$a^{-1} + aa^{-1}=-a $$
		Simplifiying, I get 
		$$a^{-1}(1+a)=-a$$
		and dividing by $(1+a)$ on both sides I get
		$$a^{-1} = \frac{-a}{(1+a)}$$
	Now to show that $a^{-1}\neq -1$, I will show $a^{-1}=-1$ cannot be true. If $a^{-1} = -1$, then 
	$$a^{-1} = \frac{-a}{(1+a)} = -1 $$
	it follows that 
	$$-a = -1(1+a) = (-1-a)$$
	and
	$$0 = -1$$
	Which is false. This shows that $a^{-1} \neq -1$. Since every element in $\mathbb{R}-\{-1\}$ has an inverse not equal to $-1$, this shows that every element in $\mathbb{R}-\{-1\}$ has an inverse.
\item[(e)] $\mathbb{R}-\{-1\}$ is a group under $*$ because it satisfies the properties of, closure, associativity, every element having and inverse and $*$ has an identity.
	\end{itemize}
\end{problem}
\begin{problem}{8}
	Conisder the binary operation $*$ on the set of matrices
	$$\mathnormal{T} = \left\{ \begin{bmatrix}a&0\\0&b\end{bmatrix} \bigg| \ a \in \mathbb{R}, \ b \in \mathbb{Z}\right\}$$
	defined as 
	$$ \begin{bmatrix} a&0\\0&b \end{bmatrix} *\begin{bmatrix} c&0\\0&d \end{bmatrix} 
	= \begin{bmatrix} ac&0\\0&b+d \end{bmatrix}.$$
	Show that $\langle T,* \rangle$ is a group.
	\\ \\
	To show that $\langle T,* \rangle$ is associative I will show that $(A*B)*C = A*(B*C)$ where $A,B,C \in T$. Since,
	$$ (A*B)*C = \begin{bmatrix} ac&0\\0&b+d\end{bmatrix} * \begin{bmatrix} e&0\\0&f \end{bmatrix} = \begin{bmatrix} ace&0\\0&b+d+f \end{bmatrix}$$
	and
	$$ A*(B*C) =A*\begin{bmatrix} ce&0\\0&d+f \end{bmatrix} = \begin{bmatrix} cea&0\\0&d+f+b \end{bmatrix}$$
Since addition is associative in $\mathbb{Z}$ and multiplication is associative in $\mathbb{R}$ it is clear that 
$$\begin{bmatrix} ace&0\\0&b+d+f \end{bmatrix} =   \begin{bmatrix} cea&0\\0&d+f+b \end{bmatrix}$$
Therefore $(A*B)*C = A*(B*C)$. This shows that $*$ is associative.
\\ \\
To show that $\langle T,* \rangle$ has an identity element, I will show that there is an element, $e \in T$, such that $A*e = A$. It follows that 
$$ A*e = \begin{bmatrix} a&0\\0&b\end{bmatrix} * e = \begin{bmatrix} a&0\\0&b \end{bmatrix} * \begin{bmatrix} e_1&0\\0&e_2 \end{bmatrix} = \begin{bmatrix} ae_1&0\\0&b+e_2 \end{bmatrix}=A.$$
Since,
$$\begin{bmatrix} ae_1&0\\0&b+e_2 \end{bmatrix}=A$$
If I let $e_1 = 1$ and $e_2 =0$, then
$$ \begin{bmatrix} 1a&0\\0&b+0 \end{bmatrix}= \begin{bmatrix} a&0\\0&b \end{bmatrix}=A$$
Since $e_1\in \mathbb{R}$ and $e_2\in \mathbb{Z}$, the identity element is 
$$ e = \begin{bmatrix} 1&0\\0 &0\end{bmatrix}$$
\\ \\
To show that $\langle T,* \rangle$ has an inverse for all elements in $T$, I show that for all $A\in T$ that there is an $A^{-1} \in T$ such that $A*A^{-1} = e$. This can be rewritten as 
$$ A*A^{-1}= \begin{bmatrix} a&0\\0&b\end{bmatrix} * \begin{bmatrix}a_1^{-1}&0\\0&a_2^{-1} \end{bmatrix}=\begin{bmatrix} aa_1^{-1}&0\\0&b + a_2^{-1} \end{bmatrix}=\begin{bmatrix} 1&0\\0&0 \end{bmatrix}.$$
So, I need to find $a_1^{-1}$ and $a_2^{-1}$ such that $aa_1^{-1}=1$ and $b+a_2^{-1} = 0$. Solving for $a_1^{-1}$, I get $a_1^{-1} = \frac{1}{a}$ and solving for $a_2^{-1}$, I get $a_2^{-1} = -b$. Since $\frac{1}{a} \in \mathbb{R}^*$ because $a\in\mathbb{R}^*$ and $-b \in \mathbb{Z}$ because $b \in \mathbb{Z}$, the inverse must exist for all elements in $T$ as
$$A^{-1} = \begin{bmatrix}\frac{1}{a}&0\\0&-b\end{bmatrix}.$$
\\ \\
Since $\langle T,* \rangle$ satasfies all properties of a group, $\langle T,* \rangle$ is a group.
\end{problem}

\begin{problem}{9}
	Let $\langle G, * \rangle$ be a group, and $a,b,c \in G$. Solve the equation $a*x*b = c$ for $x$.
	\\ \\
	Since $a,b,c \in G$, $G$ is closed, and $a$ and $b$ have left and right inverses, by left multiplying the L.H.S by $a^{-1}$ and right multiplying by $b^{-1}$ I get
	$$ a^{-1}a*x*bb^{-1} = a^{-1}cb^{-1}$$
	Therefore $x = a^{-1}cb^{-1}$
\end{problem}

\begin{problem}{10}
	Let $\langle G,* \rangle$ be a group. Use induction to show that, for any integer $n\geq 2$, and for any $n$ elements $a_1,a_2,...,a_n$ from $G$,
	$$(a_1*a_2*...*a_n)' = a'_n*a'_{n-1}*...*a'_1$$
	\\ \\
	\begin{proof}
	Using the steps of mathematical induction, I show that this holds for the base case $n=2$. That is I need to show that
	$$(a_1 * a_2)' = a'_2 * a'_1 $$
	Since $\langle G,* \rangle$ is a group, it is closed, $*$ is associative, and each element has an inverse. If I multiply both sides by $(a_1*a_2)$, I get
	$$ e = a'_2 * a'_1 * (a_1 * a_2)$$
	Using the associative property, I get
	$$ e =  a'_2 * (a'_1 * a_2 * a_1) = a'_2 * (e * a_2) = a'_2 * a_2 = e$$
	This shows that $(a_1 * a_2)' = a'_2 * a'_1$.\\ \\
	 Now, I assume this holds for $n$, that is I want to show that
	  $$ (a_1*a_2*...*a_n)' = a'_{n} *a'_{n-1} ... * a'_1$$
	  and I want to show that this holds for $n+1$. That is, 
	$$(a_1*a_2*...*a_n*a_{n+1})' = a'_{n+1} * a'_{n} * ... * a'_1.$$
	It follows from the IHOP, associativity, and inverse multiplication that
	$$e=(a_1*a_2*...*a_{n+1}) * a'_{n+1} * a'_{n} * ... * a'_1 $$
	it then follows that
	$$ e=(a_1*a_2*...*a_{n}*a_{n+1} * a'_{n+1}) * a'_{n} * ... * a'_1 = (a_1*a_2*...*a_{n}) * a'_{n} * ... * a'_1 $$
	Finally, when multiplying by the inverse of $(a_1 * a_2 *...* a_n)$ on both sides I get
	$$(a_1 * a_2 *... * a_n)' = a'_1 * a'_2 *... a'_n.$$
	This shows that the identity still holds whith $n+1$, and the induction is completed.
	\end{proof}
	
\end{problem}
\end{document}
