%!TeX program=lualatex
\documentclass[titlepage]{article}
 \usepackage[table]{xcolor}
\usepackage{../Head}



\frenchspacing
\begin{document}

% --------------------------------------------------------------
%                         Start here
% --------------------------------------------------------------
 
%\maketitle
%\clearpage
\fancyhf{}
\fancyhead[RO,R]{Abstract I}
\fancyhead[LO,L]{Dakota Wicker}
\fancyhead[CO,C]{Homework XI}
\cfoot{\thepage}

\begin{problem}{1}
Let $G$ be an abelian group of order 1200.
\begin{itemize}
\item[(a)] List its isomorphism classes in the form $\bigoplus\limits_{i=1}^k{\Z_{m_i}}$ where each $m_i$  is a prime-power
\item[(b)] List its isomorphism classes in the form $\bigoplus\limits_{i=1}^k{\Z_{m_i}}$ in which $m_{i+1} \ | \ m_i$ for $1 \leq i \leq k-1$
\item[(c)] If $G$ contains an element $a$ of order at least 300, what could it be isomporphic to? Explain! Express the answers in the form specified in (b).
\end{itemize}
\end{problem}
\begin{solution}
\begin{itemize}
\vspace{-5mm}
\item[(a)] Since $|G| =1200 = 2^4 \cdot 3^1 \cdot 5^2,$ the isomorphism classes are: 
\begin{itemize}
\item[\cdot] $\Z_{16} \oplus \Z_3 \oplus \Z_{25}$
\item[\cdot] $\Z_{16} \oplus \Z_3 \oplus \Z_5 \oplus \Z_5$
\item[\cdot] $\Z_{8} \oplus \Z_2 \oplus \Z_3 \oplus \Z_{25}$
\item[\cdot] $\Z_8 \oplus \Z_2 \oplus \Z_3 \oplus \Z_5 \oplus \Z_5$
\item[\cdot] $\Z_4 \oplus \Z_4 \oplus \Z_3 \oplus \Z_{25}$
\item[\cdot] $\Z_4 \oplus \Z_4 \oplus \Z_3 \oplus \Z_5 \oplus \Z_5$
\item[\cdot] $\Z_4 \oplus \Z_2 \oplus \Z_2 \oplus \Z_3 \oplus \Z_5 \oplus \Z_5$
\item[\cdot] $\Z_4 \oplus \Z_2 \oplus \Z_2 \oplus \Z_3 \oplus \Z_{25}$
\item[\cdot] $\Z_2 \oplus \Z_2 \oplus  \Z_2 \oplus \Z_2 \oplus \Z_3 \oplus \Z_{25}$
\item[\cdot] $\Z_2 \oplus \Z_2 \oplus  \Z_2 \oplus \Z_2 \oplus \Z_3 \oplus \Z_5 \oplus \Z_5$
\end{itemize}
\item[(b)]
\begin{itemize}
\item[\cdot] $\Z_{1200}$
\item[\cdot] $\Z_{600} \oplus \Z_2$
\item[\cdot] $\Z_{300} \oplus \Z_4$
\item[\cdot] $\Z_{300} \oplus \Z_2 \oplus \Z_2$
\item[\cdot] $\Z_{240} \oplus \Z_5$
\end{itemize}
\item[(c)] If $G$ contains an element of order at least 300, that means that the order of each group in the isomorphism classes must be at least 300. This is when $G \cong \Z_{16} \oplus \Z_3 \oplus \Z_{25},$ and $G \cong \Z_4 \oplus \Z_4 \oplus \Z_3 \oplus \Z_{25}$.
\end{itemize}
\end{solution}

\begin{problem}{2}
How many isomorphism classes are there for an abelian group of order $p^6$, where $p$ is a prime? List them explicitly.
\end{problem}
\begin{solution}
The isomorphism classes for an abelian group of order $p^6$ are, $\Z_{p^6}, \Z_{p^5}\oplus\Z_p, \Z_{p^4}\oplus \Z_{p^2},$ and $\Z_{p^3} \oplus \Z_{p^3}.$
\end{solution}
\begin{problem}{3}
Consider the subgroup $H = \left\langle \begin{bmatrix}1&4\\0&1 \end{bmatrix}\right\rangle$ of the group.
$$G = \left\{ \begin{bmatrix}1&k \\ 0&1\end{bmatrix} \ \bigg| \ k \in \Z_{24}\right\}.$$
Describe its left cosets by naming them with proper notation, and listing their elements explicitly.
\end{problem}
\begin{solution}
It can be shown that $G \cong \Z_{24}$ so $|G| = 24$ and $\text{ord}_{\Z_{24}}(\langle4\rangle) = 6 = \text{ord}_{G}(\langle [\begin{smallmatrix}1&4 \\ 0 & 1 \end{smallmatrix}]\rangle)$. Using this, I know that the index, $(G:H) = \frac{|G|}{|H|} = \frac{24}{6} = 4$. Those 4 elements are:
$$[\begin{smallmatrix} 1 & 0\\ 0 & 1 \end{smallmatrix}]H = \{ [\begin{smallmatrix} 1 & 0 \\ 0 & 1 \end{smallmatrix}],   [\begin{smallmatrix} 1 & 4 \\ 0 & 1 \end{smallmatrix}],   [\begin{smallmatrix} 1 & 8 \\ 0 & 1 \end{smallmatrix}],   [\begin{smallmatrix} 1 & 12 \\ 0 & 1 \end{smallmatrix}],   [\begin{smallmatrix} 1 & 16 \\ 0 & 1 \end{smallmatrix}],   [\begin{smallmatrix} 1 & 20 \\ 0 & 1 \end{smallmatrix}]\},$$   $$[\begin{smallmatrix} 1 & 1\\ 0 & 1 \end{smallmatrix}]H = \{ [\begin{smallmatrix} 1 & 1 \\ 0 & 1 \end{smallmatrix}],   [\begin{smallmatrix} 1 & 5 \\ 0 & 1 \end{smallmatrix}],   [\begin{smallmatrix} 1 & 9 \\ 0 & 1 \end{smallmatrix}],   [\begin{smallmatrix} 1 & 13 \\ 0 & 1 \end{smallmatrix}],   [\begin{smallmatrix} 1 & 17 \\ 0 & 1 \end{smallmatrix}],   [\begin{smallmatrix} 1 & 21 \\ 0 & 1 \end{smallmatrix}]\},$$   $$[\begin{smallmatrix} 1 & 2\\ 0 & 1 \end{smallmatrix}]H = \{ [\begin{smallmatrix} 1 & 2 \\ 0 & 1 \end{smallmatrix}],   [\begin{smallmatrix} 1 & 6 \\ 0 & 1 \end{smallmatrix}],   [\begin{smallmatrix} 1 & 10 \\ 0 & 1 \end{smallmatrix}],   [\begin{smallmatrix} 1 & 14 \\ 0 & 1 \end{smallmatrix}],   [\begin{smallmatrix} 1 & 18 \\ 0 & 1 \end{smallmatrix}],   [\begin{smallmatrix} 1 & 22 \\ 0 & 1 \end{smallmatrix}]\},$$   $$[\begin{smallmatrix} 1 & 3\\ 0 & 1 \end{smallmatrix}]H = \{ [\begin{smallmatrix} 1 & 3 \\ 0 & 1 \end{smallmatrix}],   [\begin{smallmatrix} 1 & 7 \\ 0 & 1 \end{smallmatrix}],   [\begin{smallmatrix} 1 & 11 \\ 0 & 1 \end{smallmatrix}],   [\begin{smallmatrix} 1 & 15 \\ 0 & 1 \end{smallmatrix}],   [\begin{smallmatrix} 1 & 19 \\ 0 & 1 \end{smallmatrix}],   [\begin{smallmatrix} 1 & 23 \\ 0 & 1 \end{smallmatrix}]\}.$$     
\end{solution}

\begin{problem}{4}
List the cosets of $H = \langle(4,2),(2,3)\rangle$ in $\Z_6 \oplus \Z_4.$ Display the elements in the lexicographic or dictionary order. Be sure to use proper notation.
\end{problem}
\begin{solution}
\vspace{-5mm}
$$H = \{(4,2),(2,0),(0,2),(4,0),(2,2),(0,0),(2,3),(0,1),(4,3),(2,1),(0,3),(4,1)\}$$ and there is one other coset which is:
$$(1,0)H = \{(5,2),(3,0),(1,2),(5,0),(3,2),(1,0),(3,3),(1,1),(5,3),(3,1),(1,3),(5,1)\}.$$
\end{solution}

\begin{problem}{5}
Find the left cosets of the subgroup $K = \{ R_0,F_1\}$ in $D_4$. Write each coset with $R_i$ first if it can be found in the coset. Arrange the elements in the Cayley table of $D_4$ according to the order in which they appear in the cosets, and shade the table as it was done in the textbook.
\end{problem}
\begin{solution}
The order of a dihedral group $D_n$ is $2n$. So the order of $D_4$ is 8. Also, $F_1$ generates $K$ with order 2. So, $(D_4 : K) = \frac{8}{2} = 4$. So there are 4 left cosets. They are represented as so.
\\
\begin{table}[h]
\centering\scalebox{1}{\begin{tabular}{|c| c| c| c| c| c| c| c| c|}\hline
    \cellcolor{gray!20}$\circ$ &\cellcolor{gray!20} $R_0$ &\cellcolor{gray!20} $F_1$ &\cellcolor{gray!20} $R_{90}$ &\cellcolor{gray!20} $E_2$ &\cellcolor{gray!20} $R_{180}$ &\cellcolor{gray!20} $F_2$ &\cellcolor{gray!20} $R_{270}$ &\cellcolor{gray!20} $E_1$  \\
    \hline
   \cellcolor{gray!20} $R_0$ & \cellcolor{red!20}$R_0$ & $\cellcolor{red!20}F_1$ & \cellcolor{green!20}$R_{90}$ & \cellcolor{green!20}$E_2$ & \cellcolor{blue!20}$R_{180}$ & \cellcolor{blue!20}$F_2$&\cellcolor{yellow!20} $R_{270}$ &\cellcolor{yellow!20} $E_1$\\\hline
   \cellcolor{gray!20} $F_1$ &\cellcolor{red!20} $F_1$& \cellcolor{red!20}$R_0$ & \cellcolor{yellow!20}$E_1$ & \cellcolor{yellow!20}$R_{270}$ & \cellcolor{blue!20}$F_2$ &\cellcolor{blue!20} $R_{180}$ & \cellcolor{green!20}$E_2$ & \cellcolor{green!20}$R_{90}$ \\\hline
   \cellcolor{gray!20} $R_{90}$ & \cellcolor{green!20}$R_{90}$ & \cellcolor{green!20}$E_2$ & \cellcolor{blue!20}$R_{180}$ &\cellcolor{blue!20} $F_2$ & \cellcolor{yellow!20}$R_{270}$ &\cellcolor{yellow!20}$E_1$ &\cellcolor{red!20} $R_0$ & \cellcolor{red!20}$F_1$\\\hline
    \cellcolor{gray!20}$E_2$ & \cellcolor{green!20}$E_2$ & \cellcolor{green!20}$R_{90}$ & \cellcolor{red!20}$F_1$ & \cellcolor{red!20}$R_0$ &\cellcolor{yellow!20} $E_1$ &\cellcolor{yellow!20}$R_{270}$ & \cellcolor{blue!20}$F_2$ & \cellcolor{blue!20}$R_{180}$\\\hline
    \cellcolor{gray!20}$R_{180}$ & \cellcolor{blue!20}$R_{180}$ &\cellcolor{blue!20} $F_2$ & \cellcolor{yellow!20}$R_{270}$&\cellcolor{yellow!20} $E_1$ &\cellcolor{red!20} $R_0$ &\cellcolor{red!20} $F_1$ & \cellcolor{green!20}$R_{90}$ &\cellcolor{green!20} $E_2$\\\hline
    \cellcolor{gray!20}$F_2$ & \cellcolor{blue!20}$F_2$ &\cellcolor{blue!20} $R_{180}$ &\cellcolor{green!20} $E_2$ &\cellcolor{green!20} $R_{90}$ & \cellcolor{red!20}$F_1$ & \cellcolor{red!20}$R_0$ &\cellcolor{yellow!20} $E_1$ & \cellcolor{yellow!20}$R_{270}$ \\\hline
    \cellcolor{gray!20}$R_{270}$ & \cellcolor{yellow!20}$R_{270}$ &\cellcolor{yellow!20} $E_1$& \cellcolor{red!20}$R_0$& \cellcolor{red!20}$F_1$& \cellcolor{green!20}$R_{90}$& \cellcolor{green!20}$E_2$& \cellcolor{blue!20}$R_{180}$&\cellcolor{blue!20} $F_2$\\\hline
    \cellcolor{gray!20}$E_1$& \cellcolor{yellow!20}$E_1$&\cellcolor{yellow!20} $R_{270}$&\cellcolor{blue!20} $F_2$& \cellcolor{blue!20}$R_{180}$&\cellcolor{green!20} $E_2$& \cellcolor{green!20}$R_{90}$& \cellcolor{red!20}$F_1$& \cellcolor{red!20}$R_0$\\\hline
\end{tabular}}\end{table}\clearpage
\end{solution}

\begin{problem}{6}
Find the right cosets of the subgroup $K = \{ R_0,F_1\}$ in $D_3$. Write each coset with $R_i$ first if it can be found in the coset. Arrange the elements in the Cayley table of $D_3$ according to the order in which they appear in the cosets, and shade the table as it was done in the textbook.
\end{problem}
\begin{solution}
The order of a dihedral group $D_n$ is $2n$. So the order of $D_3$ is 6. Also, $F_1$ generates $K$ with order 2. So, $(D_3 : K) = \frac{6}{2} = 3$. So there are 3 left cosets. They are represented as so.\begin{table}[h]
\centering\scalebox{1}{\begin{tabular}{|c| c| c| c| c| c| c|}
\hline
    \cellcolor{gray!20}$\circ$ & \cellcolor{gray!20}$R_0$ &\cellcolor{gray!20} $F_1$ &\cellcolor{gray!20} $R_{120}$ &\cellcolor{gray!20} $F_3$ &\cellcolor{gray!20} $R_{240}$ &\cellcolor{gray!20} $F_2$  \\
    \hline
    \cellcolor{gray!20}$R_0$ & \cellcolor{red!20}$R_0$ &\cellcolor{red!20} $F_1$ & \cellcolor{green!20}$R_{120}$ & \cellcolor{green!20}$F_3$ & \cellcolor{blue!20}$R_{240}$ &\cellcolor{blue!20} $F_2$\\\hline
    \cellcolor{gray!20}$F_1$ & \cellcolor{red!20}$F_1$ &\cellcolor{red!20} $R_{0}$ &\cellcolor{green!20} $F_3$ & \cellcolor{green!20}$R_{120}$ &\cellcolor{blue!20} $F_2$ & \cellcolor{blue!20}$R_{240}$ \\\hline
    \cellcolor{gray!20}$R_{120}$ & \cellcolor{green!20}$R_{120}$ & \cellcolor{blue!20}$F_2$ & \cellcolor{blue!20}$R_{240}$ &\cellcolor{red!20} $F_1$ & \cellcolor{red!20}$R_0$ &\cellcolor{green!20}$F_3$\\\hline
    \cellcolor{gray!20}$F_3$ & \cellcolor{green!20}$F_3$ &\cellcolor{blue!20} $R_{240}$ & \cellcolor{blue!20}$F_2$ & \cellcolor{red!20}$R_0$ &\cellcolor{red!20} $F_1$ &\cellcolor{green!20}$R_{120}$\\\hline
    \cellcolor{gray!20}$R_{240}$ & \cellcolor{blue!20}$R_{240}$ &\cellcolor{green!20} $F_3$ & \cellcolor{red!20}$R_0$& \cellcolor{blue!20}$F_2$ & \cellcolor{green!20}$R_{120}$ & \cellcolor{red!20}$F_1$\\\hline
    \cellcolor{gray!20}$F_2$ & \cellcolor{blue!20}$F_2$ &\cellcolor{green!20} $R_{120}$ & \cellcolor{red!20}$F_1$ & \cellcolor{blue!20}$R_{240}$ & \cellcolor{green!20}$F_3$ & \cellcolor{red!20}$R_0$ \\\hline
\end{tabular}}\end{table}
\end{solution}

\begin{problem}{7}
Find the index of $\langle 3 \rangle$ in the group $\Z_{24}$. How many different cosets does $\langle 3 \rangle$ have?
\end{problem}
\begin{solution}
The order of $\Z_{24}$ is 24 and $\langle 3 \rangle = \{3,6,9,12,15,18,21,0\}$, so $|\langle 3 \rangle| = 8.$ Therefore the index of $\langle 3 \rangle$ is $(\Z_{24} : \langle 3 \rangle) = \frac{24}{8} = 3$. This means that there are 3 left cosets. Since $\Z_{24}$ is abelian, $a\langle3\rangle = \langle3\rangle a, \ \forall a \in \Z_{24}.$ So, $\langle 3 \rangle$ only has 3 different cosets.
\end{solution}
\begin{problem}{8}
Let $\sigma = (1,2,5)(2,3)$ in $S_5$. Find the index of $\langle \sigma \rangle$ in $S_5$.
\end{problem}
\begin{solution}
$|\langle(1,2,5)(2,3)\rangle| = \big|\langle(\begin{smallmatrix}1 & 2 & 3 & 4 & 5 \\ 2 & 3 & 5 & 4 & 1 \end{smallmatrix})\rangle\big| =\big|(\begin{smallmatrix}1 & 2 & 3 & 4 & 5 \\ 2 & 3 & 5 & 4 & 1 \end{smallmatrix})\big| = \text{LCM}(4,1) = 4$ and $|S_5| = 5!$, so the index of $|\langle(1,2,5)(2,3)\rangle|$ is $(S_5 : \langle(1,2,5)(2,3)\rangle) = \frac{120}{4} = 30$.
\end{solution}

\begin{problem}{9}
Let $H$ be a normal subgroup of $G$, and let $m = (G:H).$ Show that $\forall a \in G, \ a^m \in H$.
\end{problem}
\begin{solution}
Since $H$ is a normal subgroup it forms the factor group $G/H$. The index is its order which is $m$. So, $(aH)^m = a^mH = H$. Using theorem 10.1.4, it is true that $aH = H$ iff $a \in H$. Since I have shown that $a^mH=H, $ this shows that $a^m \in H$.
\end{solution}

\begin{problem}{10}
Evaluate the order of the element $26 + \langle 12 \rangle$ in the factor group $\Z_{60}/\langle 12 \rangle$.
\end{problem}
\begin{solution}
$\langle 12 \rangle = \{0,12,24,36,48\}$ and $26 + \langle 12 \rangle = \{26,38,50,2,14\}.$ I need to find an $n$ such that $n26 = H$. I can eliminate sets easily if they do not share any elements with $H$ because if so, they cannot possibly be $H$. So, $2\cdot26 \text{(mod 60)} \equiv 52$ which is not in $H$, $3 \cdot 26 \text{(mod 60)} \equiv 18$ which is not in $H$, $4 \cdot 26 \text{(mod 60)} \equiv 44$ which is not in $H$, and $5 \cdot 26 \text{(mod 60)} \equiv 10$ which is not in $H$, but $6 \cdot 26 \text{(mod 60)} \equiv 36$ which is in $H$, so checking to see if all elements are the same I get $6\cdot26 = \{36,48,0,12,24\}$. Therefore $6$ is the order of $26 + \langle 12 \rangle$. 
\end{solution}

\begin{problem}{11}
Find the order of the factor group $(\Z_{12} \oplus \Z_{18})/\langle(4,3)\rangle$.
\end{problem}
\begin{solution}
The order of a factor group is its index. The order of $\Z_{12}\oplus \Z_{18}$ is equal to $|\Z_{12}| \cdot |\Z_{18}| = 216.$ $\langle(4,3)\rangle = \{(4,3), (8,6), (0,9), (4,12), (8,15), (0,0)\}$, so the order of $\langle (4,3) \rangle$ is 6. Hence, the index is $(\Z_{12}\oplus \Z_{18} : \langle (4,3) \rangle) = \frac{216}{6} = 36$ so, $|(\Z_{12}\oplus \Z_{18})/\langle(4,3)\rangle| = 36.$
\end{solution}

\begin{problem}{12}
Let $H$ be a normal subgroup of $G$, and let $a \in G$. If the coset $aH$ has order 3 in the factor group $G/H$, and $|H| = 10$, what are the possible orders of $a$ in $G$?
\end{problem}
\begin{solution}
Since $aH$ has order 3, $a^3H = H$. So, $a^3 \in H$. Since the order of $H$ is 10, $h^{10} = e$. By corollary 10.2.4, it follows that $(a^3)^{10} = a^{30} = e$. So, $|a| \ | \ 30$ is true and 1 and 2 are not orders because $|aH|=3$. So, $|a| \in \{3,5,6,10,15,30\}$.
\end{solution}
\end{document}