\documentclass[12pt]{article}
 
\usepackage[margin=1in]{geometry} 
\usepackage{amsmath,amsthm,amssymb}

 
\newenvironment{theorem}[2][Theorem]{\begin{trivlist}
\item[\hskip \labelsep {\bfseries #1}\hskip \labelsep {\bfseries #2.}]}{\end{trivlist}}
\newenvironment{lemma}[2][Lemma]{\begin{trivlist}
\item[\hskip \labelsep {\bfseries #1}\hskip \labelsep {\bfseries #2.}]}{\end{trivlist}}
\newenvironment{exercise}[2][Exercise]{\begin{trivlist}
\item[\hskip \labelsep {\bfseries #1}\hskip \labelsep {\bfseries #2.}]}{\end{trivlist}}
\newenvironment{problem}[2][Problem]{\begin{trivlist}
\item[\hskip \labelsep {\bfseries #1}\hskip \labelsep {\bfseries #2.}]}{\end{trivlist}}
\newenvironment{question}[2][Question]{\begin{trivlist}
\item[\hskip \labelsep {\bfseries #1}\hskip \labelsep {\bfseries #2.}]}{\end{trivlist}}
\newenvironment{corollary}[2][Corollary]{\begin{trivlist}
\item[\hskip \labelsep {\bfseries #1}\hskip \labelsep {\bfseries #2.}]}{\end{trivlist}}
 
\begin{document}
 
% --------------------------------------------------------------
%                         Start here
% --------------------------------------------------------------
 
\title{Homework 1}%replace X with the appropriate number
\author{Dakota Wicker\\ %replace with your name
Abstract Algebra I} %if necessary, replace with your course title
 
\maketitle
 
\begin{problem}{1} %You can use theorem, exercise, problem, or question here.  Modify x.yz to be whatever number you are proving
Prove that if x is rational, and y is irrational, then $x + y$ is also irrational.
\end{problem}
 
\begin{proof}
Suppose $x+y$ is rational, then $x+y$ can be written as
$$x + y = \frac{p}{q} \quad  p,q \in \mathbb{Z}, \quad q\neq 0.$$
Since x is also rational, then x can be written as 
$$x = \frac{a}{b} \quad a,b \in \mathbb{Z}, \quad b\neq 0.$$
Substituting, this can be rewritten as
$$\frac{a}{b} + y = \frac{p}{q}$$
With further algebraic manipulation, this is written as
$$y = \frac{p}{q} - \frac{a}{b} = \frac{pb-aq}{qb}$$
Since $y$ can be written in terms of the quotient of two integers, then $y$ must be rational. This is a contradiction with the initial assumption that $y$ is irrational. Therefore $x+y$ is irrational.
%Note 1: The * tells LaTeX not to number the lines.  If you remove the *, be sure to remove it below, too.
%Note 2: Inside the align environment, you do not want to use $-signs.  The reason for this is that this is already a math environment. This is why we have to include \text{} around any text inside the align environment.
\end{proof}
 
\begin{problem}{2}
Use mathematical induction to prove that the following holds for all positive integers $$\sum_{i=1}^{n}i^3 = \frac{n^2(n+1)^2}{4}$$
\end{problem}

\begin{proof}
Using the steps of induction I first show that this holds for $n=1$
$$1^3 =  \frac{n^2(n+1)^2}{4} = \frac{1(2)^2}{4} = 1$$
Following the steps of induction, I assume this is true for some $n = k, k \ge 1$. That is,
$$\sum_{i=1}^{k} i^3 = \frac{k^2(k+1)^2}{4} $$
Now I show this works for k+1. To do this, I will show that
$$\frac{(k+1)^2(k+2)^2}{4} = \frac{k^2(k+1)^2}{4} + (k+1)^3$$
We can show this by algebraic manipulation
\begin{align}
\frac{(k+1)^2(k+2)^2}{4} &=\frac{k^2(k+1)^2}{4} + (k+1)^3 \nonumber \\&= \frac{k^2(k+1)^2}{4} + (k^3 + 3k^2 + 3k + 1)\nonumber \\&= \frac{k^4 + 2k^3 + k^2}{4} + (k^3 + 3k^2 + 3k + 1) \nonumber\\&= \frac{k^4 + 2k^3 + k^2}{4} + \frac{4k^3 + 12k^2 + 12k + 4}{4}\nonumber\\ &= \frac{k^4 + 6k^3 + 13k^2 + 12k + 4}{4}\nonumber \\ &=\frac{(k+1)^2 (k+2)^2}{4} \nonumber
\end{align}
Therefore, $$\sum_{i=1}^{n}i^3 = \frac{n^2(n+1)^2}{4}$$ is true for all positive integers.
\end{proof}
\begin{problem}{3}
Use induction to show that $2^{2n} - 1$ is divisible by 3 for all positive integers $n$.
\end{problem}
\begin{proof}
Using the steps of induction I first show that this holds for $n=1$.
$2^2 -1 = 3$ and 3 is divisible by 3. Next I assume that $$2^{2k} - 1 = 3c $$ is true for some $n=k, k \ge 1$ and $c \in \mathbb{Z}$.
Now I show this holds true for $n=k+1$
\begin{align}
2^{2(k+1)} -1 &= (2^{2k} 2^2) - 1 \nonumber \\ &= ((3c+1)2^2)-1 \nonumber \\ &= (12c + 4)-1 \nonumber \\&= 12c + 3 \nonumber \\ &= 3(4c+1) \nonumber
\end{align}
Since this relationship holds for k+1, this proves that $2^{2k}-1$ is divisible by 3.
\end{proof}
\begin{problem}{4}
Use two-column method to find the linear combination that produces the greatest common divisor of 6157 and 6419. \\ \\
\[ \begin{array}{r|r|r}
1 &6419 \quad 6157 \\ 23 &6157 \quad 6026 \\\hline
2 &262 \quad \quad 131 \\
1 & 131 \quad \quad 131 \\\hline
 & 131 \quad \quad \quad 0
\end{array}
\]
This shows that GCD(6419,6157) = 131. Calculating $s_k$ and $t_k$ I get
\[ 
\begin{array}{r|r}
s_k & t_k \\\hline
0 & 1 \\\hline
1 & 0 \\\hline
-1 & 1 \\\hline
24 & -23 \\\hline
-49 & 47 \\\hline
73 & -70
\end{array}
\]
Giving the linear combination of $6157 \cdot 73 + 6419 \cdot -70 = 131$
\end{problem}

\begin{problem}{5}
Evaluate, $by \ hand$ (hence, in the easiest way), the value of $25^4 \cdot 20^3$(mod 23). Explain how you obtain the answer by showing the intermediate steps.
\\ \\
Since, $25 \equiv 2 \ \text{(mod 23)}$ and $20 \equiv -3 \ \text{(mod 23)}$, I can rewrite the problem as finding the value of 
$$2^4 \cdot -3^3 \ \text{(mod 23).}$$
This is equivalent to
$$ 16 \cdot -27 \ \text{(mod 23)} $$
and since $16 \equiv -7 \ \text{(mod 23)}$ and $ -27 \equiv -4 \ \text{(mod 23)}$, it follows that 
$$-7 \cdot -4 \ \text{(mod 23)} = 28 \ \text{(mod 23)} = 5 $$
leaving 5 as the value of $25^4 \cdot 20^3$ (mod 23).
\end{problem}
\begin{problem}{6}
Use the two-column method to find the integers $s$ and $t$ such that $$ 101s + 7007t = 1.$$
Finding the GCD(7007,101) gives me
\[ \begin{array}{r|r|r}
69 &7007 \quad 101 \\ 2 &101 \quad \quad 38\\\hline
1 &38 \quad  \quad 25 \\
1 & 25  \quad \quad 13\\\hline
1 & 13  \quad \quad 12 \\
12 & 12 \quad \quad 1 \\\hline
 & 1 \quad \quad 0
 \end{array}
\]
 GCD(7007,101) = 1. Finding $s_k$ and $t_k$ I get
 \[ 
\begin{array}{r|r}
s_k & t_k \\\hline
0 & 1 \\\hline
1 & 0 \\\hline
-69 & 1 \\\hline
139 & -2 \\\hline
-208 & 3 \\\hline
347 & -5 \\\hline
-555 & 8 
\end{array}
\]
Therefore $s = -555, t = 8$.
\end{problem}

\begin{problem}{7}
Use the result from the last problem to solve the congruence $$101x \equiv 1 \ \text{(mod 7007)}$$
$101\cdot-555 \equiv 1$ (mod 7007). So, $x = 5$.
\end{problem}

\begin{problem}{8}
Evaluate $7007^-1$ (mod 101)
\\ \\
A modular multiplicative inverse of an integer $a$ (mod m) is an integer $x$ where  $ax \equiv 1$ (mod $m$). So, for this problem I need to find x where $7007 \cdot x \equiv 1$ (mod 101). Since 7007 (mod 101) is equivalent to 38 ($101 \cdot 69$) (mod 101), the problem is rewritten to be $38 \cdot x \equiv 1$ (mod 101). It now becomes easier to see that $38 \cdot 8 = 304 = (101 \cdot 3)+1 \equiv 1$ mod(101). Therefore, $7007^-1$ (mod 101) = 8.
\end{problem}

\begin{problem}{9}
Use repeated squaring to evaluate $12^{189}$ (mod 37).
\\ \\
\begin{align*}
	12^1\ \ &= 12 \ \text{(mod 37)}\\
	12^2\ \ &= -4 \ \text{(mod 37)}\\
	12^4\ \ &= (-4)^2 \ \text{(mod 37)} = 16 \ \text{(mod 37)}\\
	12^8\ \ &= 16^2 \ \text{(mod 37)} = -3 \ \text{(mod 37)} \\ 
	12^{16}\ &= (-3)^2 \ \text{(mod 37)} = 9 \ \text{(mod 37)}\\
	12^{32}\ &= 9^2 \ \text{(mod 37)} = 7 \ \text{(mod 37)}\\
	12^{64}\ &= 7^2 \ \text{(mod 37)} = 12 \ \text{(mod 37)}\\ 
	12^{128}\ &= 12^2 \ \text{(mod 37)} = -4 \ \text{(mod 37)}
\end{align*}
Since $12^{189} = 12^{128} \cdot 12^{32} \cdot 12^{16} \cdot 12^8 \cdot 12^4 \cdot 12^1$ this means that $12^{189} \ \text{(mod 37)} = -4 \cdot 7 \cdot 9 \cdot -3 \cdot 16 \cdot 12 = 1 \ \text{(mod 37)}$
\end{problem}

\begin{problem}{10}
Write the complex number $\frac{1+2i}{(2-3i)(3+4i)}$ in the standard form $a+bi$
\\ \\
First, I factor the denominator.
$$\frac{1 + 2i}{18-i}$$
Then, I mutliply by the conjugate.
$$\frac{1 + 2i}{18-i} \cdot \frac{18+i}{18+i} = \frac{16+37i}{325}$$
Putting this in $a + bi$ form gives me the answer.
$$ \frac{16}{325} + \frac{37i}{325}$$
\end{problem}
\end{document}