\documentclass[titlepage]{article}
 
\usepackage[margin=1in]{geometry} 
\usepackage{amsmath,amsthm,amssymb,fancyhdr,gensymb}
\pagestyle{fancy}

\newenvironment{theorem}[2][Theorem]{\begin{trivlist}
\item[\hskip \labelsep {\bfseries #1}\hskip \labelsep {\bfseries #2.}]}{\end{trivlist}}
\newenvironment{lemma}[2][Lemma]{\begin{trivlist}
\item[\hskip \labelsep {\bfseries #1}\hskip \labelsep {\bfseries #2.}]}{\end{trivlist}}
\newenvironment{exercise}[2][Exercise]{\begin{trivlist}
\item[\hskip \labelsep {\bfseries #1}\hskip \labelsep {\bfseries #2.}]}{\end{trivlist}}
\newenvironment{problem}[2][Problem]{\begin{trivlist}
\item[\hskip \labelsep {\bfseries #1}\hskip \labelsep {\bfseries #2.}]}{\end{trivlist}}
\newenvironment{question}[2][Question]{\begin{trivlist}
\item[\hskip \labelsep {\bfseries #1}\hskip \labelsep {\bfseries #2.}]}{\end{trivlist}}
\newenvironment{corollary}[2][Corollary]{\begin{trivlist}
\item[\hskip \labelsep {\bfseries #1}\hskip \labelsep {\bfseries #2.}]}{\end{trivlist}}
 
\begin{document}
 
% --------------------------------------------------------------
%                         Start here
% --------------------------------------------------------------
 
%\title{Weekly Homework II}%replace X with the appropriate number
%\author{Dakota Wicker\\ %replace with your name
%Abstract Algebra I} %if necessary, replace with your course title
%\maketitle
%\clearpage
\fancyhf{}
\fancyhead[RO,RE]{Abstract I}
\fancyhead[LO,LE]{Dakota Wicker}
\fancyhead[CO,CE]{Homework II}
\begin{problem}{1}
Express the complex number $\frac{4+7i}{2-5i}$ in polar form. First write the answer in exact form, $then$ approximate to 2 decimal places. You may leave the angle in degrees.
\\ \\
First, I start with by getting the number into $a + bi$ form.  To do this I multiply by the conjugate
$$\frac{4+7i}{2-5i} \cdot \frac{2+5i}{2+5i} = \frac{-27+34i}{29} = \frac{-27}{29} + \frac{34i}{29}$$
Now I need to get this in the exact form form $z=r(\cos{\theta} + i\sin{\theta})$. $$\theta = 180^\circ - \arg{\left(\frac{-27}{29} + \frac{34}{29}i\right)} = 180^\circ - \tan^{-1}{\left(\frac{34}{27}\right)} = 180^\circ - 51.55^\circ = 128.45^\circ $$
$$r = \sqrt{{\left(\frac{-27}{29}\right)^2} + \left(\frac{34}{29}\right)^2} = \sqrt{\frac{65}{29}}$$
Therefore $z = \sqrt{\frac{65}{29}}\cdot \text{cis}(180^\circ - \tan^{-1}{\left(\frac{34}{27}\right)}) =  1.50\cdot\text{cis}{\big(128.45^\circ\big)} = -1.40 + 5.21i$
\end{problem}
\begin{problem}{2}
Let $z=\sqrt{2}(\cos{32^\circ + i\sin{32^\circ}})$. Find the $exact \ value$ of $z^{15}$, which means you should leave your answer in terms of radical, sine and cosine functions. However, simplifiy your answer as much as possible.
\\ \\ 
By using De Moivre's Theorem:
$$z^n = (r\text{cis}(\theta))^n = r^n \text{cis}(n\theta)$$
I can find $z^{15}$ by substituting $n,\theta$ and $r$ and expanding. It follows that
\begin{align*}
z^{15} &= \sqrt{2}^{15}\cdot(\cos{(15\cdot32)} + i\sin{(15\cdot32)}) \\
&=\sqrt{2}^{15} \cdot (\cos{(480)} + i\sin{(480)})
\end{align*}
\end{problem}

\begin{problem}{3}
Approximate, in 4 decimal places, all of the 5th roots of $-17-26i$.
\\ \\
By using De Moivre's Theorem I can use the fact
$$ z^{\frac{1}{n}} = r^{\frac{1}{n}} \cdot \text{cis}\bigg(\frac{\theta}{n} + \frac{2k\pi}{n}\bigg)$$
I can find $(-17-26i)^{\frac{1}{5}}$ by substituting $r, n, k, \text{and} \ \theta$ with the corresponding values:
$$ \theta = \tan^{-1}(26/17) + \pi$$
$$r =\sqrt{(-17)^2 + (-26)^2} = \sqrt{965}$$
$$n = 5$$
$$ k = 0, 1, 2, ..., n-1 $$
Substituting, I get the roots to be
\begin{align*}
z^{\frac{1}{5}} &= \sqrt{965}^{\frac{1}{5}} \text{cis}\bigg(\frac{\tan^{-1}(\frac{26}{17})}{5} + \frac{0\cdot\pi}{5} \bigg) = \ \ 1.3467 + 1.4626i\\
z^{\frac{1}{5}} &= \sqrt{965}^{\frac{1}{5}} \text{cis}\bigg(\frac{\tan^{-1}(\frac{26}{17})}{5} + \frac{2\cdot\pi}{5}\bigg) =-0.9749 + 1.7327i\\
z^{\frac{1}{5}} &= \sqrt{965}^{\frac{1}{5}} \text{cis}\bigg(\frac{\tan^{-1}(\frac{26}{17})}{5} + \frac{4\cdot\pi}{5}\bigg) =-1.9492 - 0.3918i\\
z^{\frac{1}{5}} &= \sqrt{965}^{\frac{1}{5}} \text{cis}\bigg(\frac{\tan^{-1}(\frac{26}{17})}{5} + \frac{6\cdot\pi}{5}\bigg) =-0.2297 - 1.9748i\\
z^{\frac{1}{5}} &= \sqrt{965}^{\frac{1}{5}} \text{cis}\bigg(\frac{\tan^{-1}(\frac{26}{17})}{5} + \frac{8\cdot\pi}{5}\bigg) = \ \ 1.8072 - 0.8288i\\
\end{align*} 
\end{problem}

\begin{problem}{4}
Use the binomial theorem to expand $(\cos\theta+i\sin\theta)^5$. Use the result to express $\cos 5\theta$ and $\sin 5\theta$ in terms of $\cos\theta$ and $\sin\theta.$
\\ \\
Expanding $(\cos\theta+i\sin\theta)^5$ I get 
$$(\cos\theta+i\sin\theta)^5 = \cos^5{\theta} + 5i\sin{\theta}\cos^4{\theta} - 10\sin^2{\theta}\cos^3{\theta} - 10i\sin^3{\theta}\cos^2{\theta}+5\sin^4{\theta}\cos{\theta} +  i\sin^5{\theta}$$
Using De Moivre's Theorem
$$ (\cos\theta+i\sin\theta)^n = \cos n\theta+i\sin n\theta$$
I show that 
$$(\cos\theta+i\sin\theta)^5 = \cos 5\theta+i\sin 5\theta$$
so I can separate the real parts and imaginary parts like so
$$\cos{5\theta} = \cos^5{\theta} - 10\sin^2{\theta}\cos^3{\theta}+5\sin^4{\theta}\cos{\theta}$$
and 
$$\sin{5\theta} = 5\sin{\theta}\cos^4{\theta} - 10\sin^3{\theta}\cos^2{\theta}+ \sin^5{\theta}$$
\end{problem}
\begin{problem}{5}
Let $a$ be a fixed number. Use induction to show that
$$\begin{bmatrix} 1 & a \\ 0 & 1 \end{bmatrix}^{n}=\begin{bmatrix} 1 & an \\ 0 & 1 \end{bmatrix}$$
for all integers $ n \geq 0$.
\begin{proof}
Using the steps of induction I show that this holds for the base case $n=1$
$$\begin{bmatrix} 1 & a \\ 0 & 1 \end{bmatrix}^{1} = \begin{bmatrix} 1 & a\cdot1 \\ 0 & 1 \end{bmatrix} = \begin{bmatrix} 1 & a \\ 0 & 1 \end{bmatrix}$$
Now, I show that for any $n= k, k \geq 0$ that if this holds for $n=k$, then this also holds for $n=k+1$. To do this, I assume the induction hypothesis is true for some $n=k, k\geq 0$, then show the induction hypothesis is true for $n=k+1$. Using the induction hypothesis, I get
\\
$$
\begin{bmatrix} 1 & a \\ 0 & 1 \end{bmatrix}^{k+1} = \begin{bmatrix} 1 & a\cdot(k+1) \\ 0 & 1 \end{bmatrix}.
$$
\\ Simplifying this matrix, I get
$$
\begin{bmatrix} 1 & a \\ 0 & 1 \end{bmatrix}^{k+1} = \begin{bmatrix} 1 & a \\ 0 & 1 \end{bmatrix}^{k} \begin{bmatrix} 1 & a \\ 0 & 1 \end{bmatrix}.
$$
\\ So, I want to show that 
$$
\begin{bmatrix} 1 & a \\ 0 & 1 \end{bmatrix}^{k}  \begin{bmatrix} 1 & a \\ 0 & 1 \end{bmatrix} =  \begin{bmatrix} 1 & a\cdot(k+1) \\ 0 & 1 \end{bmatrix}.
$$
\\Multiplying out the L.H.S I get
$$
\begin{bmatrix}1 & ak \\ 0 & 1 \end{bmatrix}  \begin{bmatrix} 1 & a \\ 0 & 1 \end{bmatrix} =  \begin{bmatrix} 1 & a+ ak \\ 0 & 1 \end{bmatrix} =  \begin{bmatrix} 1 & a\cdot(k+1) \\ 0 & 1 \end{bmatrix}
$$
Therefore, this shows that induction holds for $n=k+1$. Since the base case holds and the inductive step holds, by mathematical induction 
$$\begin{bmatrix} 1 & a \\ 0 & 1 \end{bmatrix}^{n}=\begin{bmatrix} 1 & an \\ 0 & 1 \end{bmatrix}$$
 holds for all positive integers $n$.
\end{proof}

\end{problem}
\begin{problem}{6} 
Evaluate
$$\text{(a)} \begin{bmatrix}2 & 3 \\ 5 & 4\end{bmatrix}^{-1}\text{(mod 11)} \quad \quad \quad \text{(b)} \begin{bmatrix} 	1+i & 2-3i \\ 3 - i & 1 +2i \end{bmatrix}^{-1} $$
\\ \\
$$ \text{(a)}\begin{bmatrix}2 & 3 \\ 5 & 4\end{bmatrix}^{-1} \equiv \frac{1}{-7} \begin{bmatrix}\phantom{-}4 & -3 \\ -5 & \phantom{-}4 \end{bmatrix} \equiv (-7)^{-1} \begin{bmatrix}\phantom{-}4 & -3 \\ -5 & \phantom{-}4 \end{bmatrix}
$$ $$ \equiv 3  \begin{bmatrix}\phantom{-}4 & -3 \\ -5 & \phantom{-}4 \end{bmatrix} \equiv \begin{bmatrix}1 & 2 \\ 7 & 6 \end{bmatrix}\text{(mod 11)} $$
\\
$$\text{(b)} \begin{bmatrix} 1+i & 2-3i \\ 3 - i & 1 + 2i \end{bmatrix} ^{-1} = \frac{1}{-4+14i}\begin{bmatrix}1 +2i & -2 + 3i \\ -3 + i & 1 + i \end{bmatrix}$$
\end{problem}
\begin{problem}{7}
Find the cube roots of the matrix $\begin{bmatrix} -1 & -3 \\ 3 & -1\end{bmatrix}$ in 3 decimal places.
\\
Since $\begin{bmatrix} -1 & -3 \\ 3 & -1\end{bmatrix}$  is isomorphic to $-1 + 3i$, I can find the cube roots of $-1 + 3i$ and show the matrix representation of those roots so the problem becomes a lot easier. To solve the for the cube roots of this complex number I will use the property
$$z^{\frac{1}{n}} = r^{\frac{1}{n}}\text{cis}\bigg(\frac{\theta}{n} + \frac{2k\pi}{n}\bigg).$$
So if I $\text{let}\ z = -1 + 3i,\ \text{then}\ \theta = arg(z) = \pi - \tan^{-1}(3)  \ \text{and} \ r = \|{z}\| = \sqrt{10}$. Then, substituting the values I get
$$ z^{\frac{1}{3}} = \sqrt[3]{\sqrt{10}} \ \text{cis}\bigg(\frac{\pi - \tan^{-1}(3)}{3} + \frac{2k\pi}{3}\bigg), \ k = 0,1,2 $$
These roots are
$$
z^{\frac{1}{3}} = 1.185 + 0.866i
$$
$$
z^{\frac{1}{3}} = -1.342 + .0594i
$$
$$
z^{\frac{1}{3}} = 0.157 + 1.459i
$$
or in matrix form
$$
z^{\frac{1}{3}} = \begin{bmatrix} 1.185 & -0.866 \\ 0.866 & \phantom{-}1.185 \end{bmatrix} 
$$
$$
z^{\frac{1}{3}} = \begin{bmatrix}-1.342 & -.0594 \\ \phantom{-}.0594 & -1.342 \end{bmatrix}
$$
$$
z^{\frac{1}{3}} = \begin{bmatrix} 0.157 & -1.459 \\ 1.459 & \phantom{-}0.157$$ \end{bmatrix}
$$
\end{problem}  
	
\begin{problem}{9}
Complete the following transformation and operation tables for $D_4$:
\\ \\
\begin{tabular}{lllll}
 & 1  & 2  & 3  & 4  \\\hline
$R_0$ & 1 & 2 & 3 & 4 \\
$R_{90}$ & 4 & 1 & 2 & 3 \\
$R_{180}$ & 3 & 4 & 1 & 2 \\
$R_{270}$& 2 & 3 & 4 & 1 \\
$F_1$ & 1 & 4 & 3 & 2 \\
$F_2$& 3 & 2 & 1 & 4 \\
$E_1$& 2 & 1 & 4 & 3 \\
$E_2$& 4 & 3 & 2 & 1
\end{tabular}
\end{problem}
\begin{tabular}{lllllllll}
. & $R_0$ & $R_{90}$ &$R_{180}$  &$R_{270}$ & $F_1$  &$F_2$  & $E_1$  & $E_2$  \\\hline
 $R_0$ & $R_0$  & $R_{90}$ &$R_{180}$  &$R_{270}$  		&$F_1$  &$F_2$  &$E_1$  &$E_2$  \\
$R_{90}$ & $R_{90}$ &$R_{180}$  &$R_{270}$  & $R_0$ 		&$E_2$  &$E_1$  &$F_1$  &$F_2$  \\
$R_{180}$ & $R_{180}$ &$R_{270}$  &$R_0$  & $R_{90}$ 	&$F_2$  &$F_1$  &$E_2$  & $E_1$  \\
$R_{270}$ &$R_{270}$  &$R_0$  &$R_{90}$  & $R_{180}$ 	&$E_1$  &$E_2$  &$F_2$  &$F_1$  \\
$F_1$ &$F_1$  &$E_1$  &$F_2$  &$E_2$  &$R_0$  &$R_{180}$  &$R_{90}$  &$R_{270}$  \\
$F_2$&$F_2$  &$E_2$  &$F_1$  &$E_1$  &$R_{180}$  &$R_0$  &$R_{270}$  &$R_{90}$  \\
$E_1$ &$E_1$  &$F_2$  &$E_2$  &$F_1$  &$R_{270}$  &$R_{90}$  &$R_0$  &$R_{180}$  \\
$E_2$ &$E_2$  & $F_1$ & $E_1$ &$F_2$  &$R_{90}$  &$R_{270}$  &$R_{180}$  &$R_0$ 
\end{tabular}
\begin{problem}{10}
Using the table from the last problem, compute the value of
$$R_{90}F_2R_{270}, \quad F_2R_{90}E_1, \quad E_2F_1F_2E_1 \quad \text{and}\quad R_{180}R_{90}F_1E_1R_{270}$$
\\
$$ R_{90}F_2R_{270}=R_{90}E_1 = F_1 $$ 
$$F_2R_{90}E_1= F_2F_1 = R_{180}$$
$$ E_2F_1F_2E_1 =E_2F_1R_{270}=E_2E_2 = R_0$$ 
$$ R_{180}R_{90}F_1E_1R_{270} =  R_{180}R_{90}F_1 F_1=  R_{180}R_{90}R_{0} = R_{180}R_{90}=R_{270}$$
\end{problem}
\end{document}