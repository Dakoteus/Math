%!TeX program=pdflatex
\documentclass[titlepage]{article}
 
\usepackage[margin=1in]{geometry} 
\usepackage{amsmath,amsthm,amssymb,fancyhdr,gensymb,arydshln,graphicx,setspace,mathtools}
\graphicspath{{C:/Users/user/Desktop/}}

\allowdisplaybreaks
\pagestyle{fancy}
\newcommand{\Mod}[1]{\ (\mathrm{mod}\ #1)}
\newenvironment{theorem}[2][Theorem]{\begin{trivlist}
\item[\hskip \labelsep {\bfseries #1}\hskip \labelsep {\bfseries #2.}]}{\end{trivlist}}
\newenvironment{lemma}[2][Lemma]{\begin{trivlist}
\item[\hskip \labelsep {\bfseries #1}\hskip \labelsep {\bfseries #2.}]}{\end{trivlist}}
\newenvironment{exercise}[2][Exercise]{\begin{trivlist}
\item[\hskip \labelsep {\bfseries #1}\hskip \labelsep {\bfseries #2.}]}{\end{trivlist}}
\newenvironment{problem}[2][Problem]{\begin{trivlist}
\item[\hskip \labelsep {\bfseries #1}\hskip \labelsep {\bfseries #2.}]}{\end{trivlist}}
\newenvironment{question}[2][Question]{\begin{trivlist}
\item[\hskip \labelsep {\bfseries #1}\hskip \labelsep {\bfseries #2.}]}{\end{trivlist}}
\newenvironment{corollary}[2][Corollary]{\begin{trivlist}
\item[\hskip \labelsep {\bfseries #1}\hskip \labelsep {\bfseries #2.}]}{\end{trivlist}}


\begin{document}

% --------------------------------------------------------------
%                         Start here
% --------------------------------------------------------------
 
%\title{Weekly Homework II}%replace X with the appropriate number
%\author{Dakota Wicker\\ %replace with your name
%Abstract Algebra I} %if necessary, replace with your course title
%\maketitle
%\clearpage
\fancyhf{}
\fancyhead[RO,RE]{Abstract I}
\fancyhead[LO,LE]{Dakota Wicker}
\fancyhead[CO,CE]{Homework VII}
\cfoot{\thepage}

\begin{problem}{1}
We have learned that 
$$ H = \left\{\begin{bmatrix} 1&0 \\ 6n & 1 \end{bmatrix} \bigg| \ n \in \mathbb{Z}\right\}$$
is a subgroup of $SL(2,\mathbb{R})$. Is $H$ cyclic? What are its generators?
\\
\\
$H$ is cyclic because there is an element $a$ that generates all elements in $H$. That is, when $n = 1$, or, $a = \left[\begin{smallmatrix} 1 & 0 \\ 6 & 1 \end{smallmatrix}\right]$. To show that this is a generator, I use the fact that 
$$\begin{bmatrix}1&0\\k&1 \end{bmatrix}^x = \begin{bmatrix}1&0\\kx & 1\end{bmatrix},\ k,x \in \mathbb{Z}.$$
If I let $6n = k$ then the resulting matrix becomes 
$$\begin{bmatrix} 1&0\\6n & 1\end{bmatrix}^x = \begin{bmatrix}1 & 0 \\ (6n)x & 1 \end{bmatrix}$$
and when $n = 1$,
$$\begin{bmatrix} 1&0\\6 & 1\end{bmatrix}^x = \begin{bmatrix}1&0\\6x & 1\end{bmatrix}.$$
Since this takes the form of any element in $H$, it is clear that $\left[\begin{smallmatrix} 1&0 \\6 & 1 \end{smallmatrix}\right]$ is the only generator for $H$.
\end{problem}

\begin{problem}{2}
Find the generator of $\langle S, \cdot \rangle$, where 
$$S = \left\{ \begin{bmatrix}1&0\\k&1 \end{bmatrix} \bigg| \ k \in \mathbb{Z}_{12} \right\}.$$
\\ 
\\
The generator of $S$ is when $k=1$. I know this because 
$$\begin{bmatrix}1&0 \\ k & 1 \end{bmatrix}^n = \begin{bmatrix} 1&0 \\ kn & 1\end{bmatrix} \text{(mod 12)}, n \in \mathbb{Z}_{12}$$
and when $k=1$ this is equivalent to 
$$\begin{bmatrix}1&0 \\ 1 & 1 \end{bmatrix}^n = \begin{bmatrix} 1&0 \\ n & 1\end{bmatrix} \text{(mod 12)}$$
So, it is clear that when $k=1$, the resulting matrix generates $S$.
\end{problem}

\begin{problem}{3}
Let $a$ be an element in a group with ord($a$) = 18. Use an appropriate formula to compute the values of $\text{ord}_{\langle a\rangle}(a^k)$, for k = 2, 3, 4, 5.
\\ \\
I use theorem 5.1.6 to get that, $|a^2| = \frac{18}{\gcd(18,2)} = \frac{18}{2} = 9$, $|a^3| = \frac{18}{\gcd(18,3)} = \frac{18}{3} = 6$, $|a^4| = \frac{18}{\gcd(18,4)} = \frac{18}{2} = 9$, and $|a^5| = \frac{18}{\gcd(18,5)} = \frac{18}{1} = 18$.
\end{problem}

\begin{problem}{4}
Let $G$ be a group with an element $a$ such that $\text{ord}_G(a) = 72$. Let $H = \langle a \rangle$.
\begin{itemize}
\item[(a)] What is the value of $|H|$?
\item[(b)] Compute $\text{ord}_H(a^6)$?
\item[(c)] Let $K = \langle a^6 \rangle.$ Determine $\text{ord}_K(a^{48})$
\end{itemize}
\hrule
\begin{itemize}
\item[(a)] Since $H$ is generated by $a$, the order of $a$ is equal to the order of the group. So, $|H| = |\langle a \rangle| = 72$
\item[(b)] Using theorem 5.1.6, I get that $|a^6| = \frac{72}{\gcd(72,6)} = \frac{72}{6} = 12$.
\item[(c)] If $a^6$ generates $K$, then $|K| = |\langle a^6 \rangle| = 12$. Using theorem 5.1.6, I get $|\langle a^{48}\rangle| = \frac{12}{\gcd(12,48)} = \frac{12}{4} = 3$.
\end{itemize}
\end{problem}

\begin{problem}{5}
Find the elements of $\mathbb{Z}_{96}$ that have order 12.
\\ \\
Since $\mathbb{Z}_{96}$ is cyclic because $\mathbb{Z}_{96} = \langle 1 \rangle$, I can use theorem 5.1.6 to show the elements of order 12. To find these elements, using theorem 5.1.6 I need to find the values of $k$ where $|1^k| = \frac{96}{\gcd(96,k)} = 12$. I find that $\gcd(96,8) = \gcd(96,40) = \gcd(96,56) = \gcd(96,88) = 8$. Since theorem 5.2.2 says that if a group has order $n$, that given a divisor $d$ of $n$, the number of elements in that group having order $d$ is $\phi(d)$, I can say that this group, $\mathbb{Z}_{96}$, with order 96 and 12 being the order which is a divisor of 96, that the amount of elements which has order 12 is $\phi(12) = 4$. So, these four elements are $\{8, 40, 56, 88\}$. 
\end{problem}
\begin{problem}{6}
Let $a$ be an element of a group $G$ such that ord($a$) = 36. Find all the elements of $H = \langle a \rangle$ with order 9.
\\ \\
Since $H = \langle a \rangle$, $|H| = |\langle a \rangle| = 36$. With this fact and theorem 5.1.6, I can find the elements of $H$ with order 9 to be $a^k$ where $a^k = \frac{36}{\gcd(36,k)} = 9$. These $k$ values are $4, 8, 16, 20, 28, 32$. So, the elements of order 9 are $a^4, a^8, a^{16}, a^{20}, a^{28}$, and $a^{32}$.
\end{problem}

\begin{problem}{7}
Let 
$$\alpha = \left(\begin{array}{ccccccc} 1 & 2 & 3 & 4 & 5 & 6 & 7 \\ 2 & 1 & 4 & 6 & 7 & 5 & 3\end{array}\right), \quad \text{and} \quad  \beta = \left(\begin{array}{ccccccc} 1 & 2 & 3 & 4 & 5 & 6 & 7 \\ 7 & 6 & 5 & 3 & 2 & 1 & 4 \end{array}\right)$$
Evaluate $\alpha^2, \beta^3 $ and $\alpha^2\beta^3.$ Express the answers in the 2-line form.
\begin{align*}
 \alpha^2 &= \left(\begin{array}{ccccccc} 1 & 2 & 3 & 4 & 5 & 6 & 7 \\ 1 & 2 & 6 & 5 & 3 & 7 & 4\end{array}\right) \\
\beta^3 &= \left(\begin{array}{ccccccc}1 & 2 & 3 & 4 & 5 & 6 & 7 \\ 3 & 7 & 6 &2 & 1 & 4 & 5 \end{array}\right) \\
 \alpha^2\beta^3 &= \left(\begin{array}{ccccccc}1 & 2 & 3 & 4 & 5 & 6 & 7 \\ 6 & 4 & 7 & 2 & 1 & 5 & 3 \end{array}\right)
 \end{align*}
 
\end{problem}

\begin{problem}{8}
Let the 2-line form of $\sigma, \gamma \in S_5$ be
$$\sigma = \left(\begin{array}{ccccc} 1 & 2 & 3 & 4 & 5 \\ 2 & 4 & 3 & 5 & 1 \end{array}\right) , \quad \gamma = \left(\begin{array}{ccccc} 1 & 2 & 3 & 4 & 5 \\ 5 & 4 & 1 & 2 & 3 \end{array}\right).$$
\begin{itemize}
\item[(a)] Find $\sigma\gamma, \gamma\sigma, \sigma^2\gamma^3\sigma,$ and $\gamma^{-1}$
\item[(b)] What are the orders of $\sigma$ and $\gamma$?
\item[(c)] Find $\sigma^{23}$
\end{itemize}
\hrule
\begin{itemize}
\item[(a)] $\sigma\gamma = \left(\begin{smallmatrix} 1 & 2 & 3 & 4 & 5 \\ 1 & 5 & 2 & 4 & 3\end{smallmatrix}\right), \ \gamma \sigma = \left(\begin{smallmatrix}1 & 2 & 3 & 4 & 5 \\ 4 & 2 & 1 & 3 & 5 \end{smallmatrix}\right), \ \sigma^2\gamma^3\sigma =  \left(\begin{smallmatrix}1 & 2 & 3 & 4 & 5 \\ 2 & 4 & 3 & 5 & 1 \end{smallmatrix}\right)^2  \left(\begin{smallmatrix}1 & 2 & 3 & 4 & 5 \\ 5 & 4 & 1 & 2 & 3 \end{smallmatrix}\right)^3  \left(\begin{smallmatrix}1 & 2 & 3 & 4 & 5 \\ 2 & 4 & 3 & 5 & 1 \end{smallmatrix}\right) =  \left(\begin{smallmatrix}1 & 2 & 3 & 4 & 5 \\ 1 & 5 & 3 & 2 & 4 \end{smallmatrix}\right), \\ \gamma^{-1} =\left(\begin{smallmatrix} 1 & 2 & 3 & 4 & 5 \\  3 & 4 & 5 & 2 & 1 \end{smallmatrix}\right), \sigma^{-1} = \left(\begin{smallmatrix} 1 & 2 & 3 & 4 & 5 \\  5 & 1 & 3 & 2 & 4 \end{smallmatrix}\right)$

\item[(b)] The order of $\sigma$ is 4 because 4 is the smallest positive integer greater than zero that gives $\iota$ when $\sigma$ is raised to the 4th power. The order of $\gamma$ is 6 because 6 is the smallest positive integer greater than zero that gives $\iota$ when $\gamma$ is raised to the 6th power.

\item[(c)] Since $\sigma$ has an order of 4, $\sigma^{20} = \iota$. It follows that $\sigma^{23} = \sigma^{20}\sigma^3 =\iota\sigma^3 = \sigma^3 = \left(\begin{smallmatrix} 1 & 2 & 3 & 4 & 5 \\  5 & 1 & 3 & 2 & 4 \end{smallmatrix}\right)$
\end{itemize}
\end{problem}
\end{document}