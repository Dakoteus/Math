%!TeX program=pdflatex
\documentclass[titlepage]{article}
 
\usepackage[margin=1in]{geometry} 
\usepackage{amsmath,amsthm,amssymb,fancyhdr,gensymb,arydshln,graphicx,setspace,mathtools}
\graphicspath{{C:/Users/user/Desktop/}}
\linespread{1.5}
\allowdisplaybreaks
\pagestyle{fancy}
\newcommand{\Mod}[1]{\ (\mathrm{mod}\ #1)}
\newenvironment{theorem}[2][Theorem]{\begin{trivlist}
\item[\hskip \labelsep {\bfseries #1}\hskip \labelsep {\bfseries #2.}]}{\end{trivlist}}
\newenvironment{lemma}[2][Lemma]{\begin{trivlist}
\item[\hskip \labelsep {\bfseries #1}\hskip \labelsep {\bfseries #2.}]}{\end{trivlist}}
\newenvironment{exercise}[2][Exercise]{\begin{trivlist}
\item[\hskip \labelsep {\bfseries #1}\hskip \labelsep {\bfseries #2.}]}{\end{trivlist}}
\newenvironment{problem}[2][Problem]{\begin{trivlist}
\item[\hskip \labelsep {\bfseries #1}\hskip \labelsep {\bfseries #2.}]}{\end{trivlist}}
\newenvironment{question}[2][Question]{\begin{trivlist}
\item[\hskip \labelsep {\bfseries #1}\hskip \labelsep {\bfseries #2.}]}{\end{trivlist}}
\newenvironment{corollary}[2][Corollary]{\begin{trivlist}
\item[\hskip \labelsep {\bfseries #1}\hskip \labelsep {\bfseries #2.}]}{\end{trivlist}}


\begin{document}
 
% --------------------------------------------------------------
%                         Start here
% --------------------------------------------------------------
 
%\title{Weekly Homework II}%replace X with the appropriate number
%\author{Dakota Wicker\\ %replace with your name
%Abstract Algebra I} %if necessary, replace with your course title
%\maketitle
%\clearpage
\fancyhf{}
\fancyhead[RO,RE]{Abstract I}
\fancyhead[LO,LE]{Dakota Wicker}
\fancyhead[CO,CE]{Homework VI}
\cfoot{\thepage}

\begin{problem}{1}
Find C($\sigma$) for each $\sigma \in D_4$. 
\begin{align*}
&C(R_0) = \{R_0, R_{90}, R_{180}, R_{270}, F_1, F_2, E_1, E_2\}\\
&C(R_{90}) = \{R_0, R_{90}, R_{180},R_{270}\}\\
&C(R_{180}) = \{R_0, R_{90}, R_{180},R_{270}, F_1, F_2, E_1, E_2\}\\
&C(R_{270}) = \{R_0, R_{90}, R_{180},R_{270}\} \\
&C(F_1) = \{R_0, R_{180}, F_1,F_2\}\\
&C(F_2) = \{R_0, R_{180}, F_1,F_2\}\\
&C(E_1) = \{R_0, R_{180}, E_1, E_2\}\\
&C(E_2) = \{R_0, R_{180}, E_1, E_2\}\\
\end{align*}
\end{problem}
\begin{problem}{2}
Show that $Z(G)$, the center of the group $G$, is a subgroup of $G$.

To show that $Z(G) \leq G$ I will use corollary 4.2.2. Since $Z(G) = \left\{g \in G \ | \ gx = xg \  \text{for all} \ x \in G \right\}$ and $ge = eg$ this means $\ e \in Z(G)$ and since $e \in G$, $Z(G)$ is a nonempty subset of $G$. To show that the inverse exists in $Z(G)$ for all elements in $Z(G)$ I use the fact that $G$ is a group, $g,g^{-1} \in G$, and $g^{-1}$ has the property that $g^{-1}g = e = gg^{-1}$. By the definition of $Z(G)$ this means that for all $g \in Z(G)$, $g^{-1} \in Z(G)$. Using the associative property, suppose $a,b \in Z(G)$. Then, $$ (ab)g = a(bg) = a(gb) = (ag)b = (ga)b = g(ab), \ \therefore ab \in Z(G)$$  
Since it has been shown that $Z(G)$ is a non-empty subset of $G$, is closed under $G$'s binary operation and there exists an inverse in $Z(G)$ for all elements in $Z(G)$, by corollary 4.2.2, $Z(G)$ is a subgroup of $G$.
\end{problem} 

\begin{problem}{3}
Prove that 
$$ G = \left\{\begin{bmatrix}1-n & -n \\ n & 1+n \end{bmatrix} \ \bigg| \ n \in \mathbb{Z} \right\} $$
is a group under matrix multiplication. Is $G$ cyclic?
\begin{proof}
To prove that $G$ is a group I will prove that it is a subgroup of $SL(2,\mathbb{Z})$. First, $G$ is a nonempty subset of $SL(2,\mathbb{Z})$ because it contains the identity element of $SL(2,\mathbb{Z})$ when $n=0$, that is, $e = \left[\begin{smallmatrix}1 & 0 \\ 0 & 1\end{smallmatrix}\right]$. To show closure, assume $A,B \in G$. Since $\det(AB) = \det(A)\cdot \det(B)$ and for all $x\in G, \det(x) = (1-n)(1+n)+n^2 = 1$, then $\det(AB) = \det(A)\cdot(B) = 1\cdot 1 = 1$. Since $SL(2,\mathbb{Z})$ contains all $2\times2$ matricies where their determinant is one, $A,B,AB\in SL(2,\mathbb{Z})$. This shows that $G$ is closed. It is also true that for all elements in $G$, the determinant is nonzero, therefore all elements have an inverse which take the form $\frac{1}{1}\left[\begin{smallmatrix}1+n & n \\ -n & 1-n \end{smallmatrix}\right]$ which is equivalent to $-1\left[\begin{smallmatrix}1-n & -n \\ n & 1+n \end{smallmatrix}\right]$ which is the element in $G$ produced by $-n$. So an inverse exists in $G$ for all elements in $G$.  Since $G$ is closed under $SL(2,\mathbb{Z})$'s binary operation, matrix multiplication, and an inverse exists for all $x\in G$, by corollary 4.2.2 $G$ is a subgroup of $SL(2,\mathbb{Z})$. Therefore $G$ is a group under matrix multiplication.
\end{proof}
$G$ is a cyclic group with a generator of $\langle\left[\begin{smallmatrix*}[r]0 & -1 \\ 1 & 2 \end{smallmatrix*}\right]\rangle$ because $\left[\begin{smallmatrix*}[r]0 & -1 \\ 1 & 2 \end{smallmatrix*}\right]^n , n \in \mathbb{Z}$ generates all elements in $G$.
\end{problem}


\begin{problem}{4}
Let $x$ be an element in a group $G$. Assume ord($x$) = 8. List the elements in $\langle x^2\rangle, \langle x^3 \rangle, \langle x^4 \rangle$ and $\langle x^5 \rangle$. Use your results to determine ord($x^2$), ord($x^3$), ord($x^4$), and ord($x^5$).
\\ \\
The elements of these orders are:
\begin{align*}
\langle x^2 \rangle &= \{x^2, x^4, x^6, e\}\\
\langle x^3 \rangle &= \{x^3, x^6, x^1, x^4, x^7, x^2, x^5, e\}\\
\langle x^4 \rangle &= \{x^4, e\}\\
\langle x^5 \rangle &= \{x^5, x^2, x^7, x^4, x^1, x^6, x^3, e\}\\
\end{align*}
Since the order of these generators is the cardinality of their set, 
\begin{align*}
\text{ord}(x^2) &= 4 \\
\text{ord}(x^3) &= 8 \\
\text{ord}(x^4) &= 2 \\
\text{ord}(x^5) &= 8 \\
\end{align*}
\end{problem}


\begin{problem}{5}
Find the order of the matrix $ \begin{bmatrix}\frac{1}{2} & \frac{\sqrt{3}}{2} \\ -\frac{\sqrt{3}}{2} & \frac{1}{2} \end{bmatrix}$ in $SL(2,\mathbb{R})$.
\\ \\
The order of the matrix is infinite. This is easy to see using the equivalent representation of this matrix, $\frac{1}{2} - \frac{\sqrt{3}}{2}i$. The order of this element is defined as the smallest positive integer $n$ such that $\left[\begin{smallmatrix}\frac{1}{2} & \frac{\sqrt{3}}{2} \\ -\frac{\sqrt{3}}{2} & \frac{1}{2} \end{smallmatrix}\right]^n = \left[\begin{smallmatrix}1 & 0\\ 0 & 1 \end{smallmatrix}\right]$or, equivalently, $(\frac{1}{2} - \frac{\sqrt{3}}{2}i)^n = 1 +0i = 1$ . Using De Moivre's theorem
$$(\cos{(\theta)} + i\sin{(\theta)})^n = \cos{(n\theta)} + i\sin{(n\theta)}$$
and expressing $\frac{1}{2}$ and $\frac{\sqrt{3}}{2}$ in terms of sine and cosine, I rewrite $(\frac{1}{2} - \frac{\sqrt{3}}{2}i)^n \ \text{as} \ \cos{(-\frac{n\pi}{3})} + i\sin{(-\frac{n\pi}{3})} $. So, solving for $$\cos{(-\frac{n\pi}{3})} + i\sin{(-\frac{n\pi}{3})} = 1$$ I get $$n = 6m, \ m\in \mathbb{Z}$$ 
Since $(\frac{1}{2} - \frac{\sqrt{3}}{2}i)^{6m} = \cos(-\frac{6m\pi}{3}) + i\sin(-\frac{6m\pi}{3}) = 1 +0i $ is equivalent to $(\frac{1}{2} - \frac{\sqrt{3}}{2}i)^0 = \cos(0) + i\sin(0) = 1 + 0i$ in the complex plane, the only solution for
 $$\begin{bmatrix}\frac{1}{2} & \frac{\sqrt{3}}{2} \\ -\frac{\sqrt{3}}{2} & \frac{1}{2} \end{bmatrix}^n =  \begin{bmatrix}1 & 0 \\ 0 & 1 \end{bmatrix} $$
 is 
 $$\begin{bmatrix}\frac{1}{2} & \frac{\sqrt{3}}{2} \\ -\frac{\sqrt{3}}{2} & \frac{1}{2} \end{bmatrix}^0 = \begin{bmatrix}1 & 0 \\ 0 & 1 \end{bmatrix}.$$ Since $n=0$ is the only solution, and $n$ is not a positive integer, the order is infinite.

\end{problem}

\begin{problem}{6}

\end{problem}

\begin{problem}{7}
Let $a$ be an element in a group with ord($a$) = 15. $Without$ explicitly listing the elements in the subgroup each $a^i$ generates, $compute$ (using a formula) the values of ord($a^3$), ord($a^6$), ord($a^9$), and ord($a^{12}$). \\ \\ Using theorem 5.1.6, $|a^k| = n/d $ where $d=\gcd(n,k)$. So
\begin{align*}
|a^3| = \frac{15}{\gcd(15,3)} &= \frac{15}{3} =5 \\
|a^6| = \frac{15}{\gcd(15,6)} &= \frac{15}{3} =5 \\
|a^9| = \frac{15}{\gcd(15,9)} &= \frac{15}{3} = 5 \\
|a^{12}|= \frac{15}{\gcd(15,12)} &= \frac{15}{3} = 5 \\
\end{align*}
are the values.
\end{problem}

\begin{problem}{8}
Consider $U_{75} = \langle \omega \rangle$, where $\omega = \text{cis}(2\pi/75)$. Compute (using an appropriate formula) the values of $\text{ord}_{U_{75}}(\omega^3), \text{ord}_{U_{75}}(\omega^5), \text{ord}_{U_{75}}(\omega^{15}), \ \text{and} \ \text{ord}_{U_{75}}(\omega^{25})$.
\\ \\ 
Since $\omega^{75} = \text{cis}(2\pi/75)^{75} = \cos(\frac{75\cdot2\pi}{75}) + i\sin(\frac{75\cdot2\pi}{75}) = \cos(2\pi) + i\sin(2\pi)=1$ and 75 is the smallest power that $\text{cis}(2\pi/75)^{75}$ can be raised to equal one, $|\omega| = 75 $. Using theorem 5.1.6,  $|\omega^k| = n/d $ where $d=\gcd(n,k)$. So
\begin{align*}
\text{ord}_{U_{75}}(\omega^3) = \frac{75}{\gcd(75,3)} &= \frac{75}{3} = 25\\
\text{ord}_{U_{75}}(\omega^5) = \frac{75}{\gcd(75,5)} &= \frac{75}{5} = 15\\
\text{ord}_{U_{75}}(\omega^{15})= \frac{75}{\gcd(75,15)} &= \frac{75}{15} = 5\\
\text{ord}_{U_{75}}(\omega^{25}) = \frac{75}{\gcd(75,25)} &= \frac{75}{25} = 3\\
\end{align*}
\end{problem}
\begin{problem}{9}
Compute the values of $\text{ord}_{\mathbb{Z}_{130}}(7)$ and $\text{ord}_{\mathbb{Z}^*_{130}}(7)$.
 \\ \\
 Since the order of $\text{ord}_{\mathbb{Z}_{130}}(7) = \text{ord}_{\mathbb{Z}_{130}}(1^7)$. I can use theorem 5.1.6 to show that $ \text{ord}_{\mathbb{Z}_{130}}(1^7) = \frac{130}{\gcd(130,7)} = \frac{130}{1} = 130 = \text{ord}_{\mathbb{Z}_{130}}(7)$. To find $\text{ord}_{\mathbb{Z}^*_{130}}(7)$, I show all elements that 7 generates.
 $$\langle7 \rangle = \{7, 49, 83, 61, 37, 129, 123, 81, 47, 69, 93, 1\}.$$
 Since all of these elements are in $\mathbb{Z}^*_{130}$, $\text{ord}_{\mathbb{Z}^*_{130}}(7) = |\langle 7 \rangle| = 12$ 
\end{problem}

\begin{problem}{10}
Show that $\mathbb{Z}^*_{25} = \langle 13 \rangle$. Use this fact to find the other generators of $\mathbb{Z}^*_{25}$
\\ \\
Since all of the numbers relatively prime to 25 are \{1,2,3,4,6,7,8,9,11,12,13,14,16,17,18,19,21,22,23,24\},  I must show that these can be expressed as $13^n$ (mod 25) where $n$ is some positive integer.
\begin{align*}
13^2 &= 19 \Mod{25}\\
13^3 &= 22 \Mod{25}\\
13^4 &= 19^2 = 11 \Mod{25} \\
13^5 &= 13^2 \cdot 13^3 = 19\cdot22 = 18 \Mod{25} \\
13^6 &= (13^3)^2 = 22^2 =  9 \Mod{25}\\
13^7 &= 13^5 \cdot 13^2 = 19\cdot18 = 17 \Mod{25} \\
13^8 &= (13^4)^2 = 21 \Mod{25} \\
13^9 &= 13^8 \cdot 13 = 21\cdot13 = 23 \Mod{25} \\
13^{10} &= (13^5)^2 = 18^2 = 24 \Mod{25}\\
13^{11} &= 13^{10} \cdot 13 = 24\cdot 13 = 12 \Mod{25} \\
13^{12} &= 13^{11}\cdot 13 =  12\cdot 13 = 6 \Mod{25} \\
13^{13} &= 13^{12} \cdot 13  = 6 \cdot 13 = 3 \Mod{25} \\
13^{14} &= 13^{13} \cdot 13 = 3 \cdot 13 = 14 \Mod{25}\\
13^{15} &= 13^{14} \cdot 13 = 14 \cdot 13 = 7 \Mod{25} \\
13^{16} &= 13^{15} \cdot 13 = 7 \cdot 13 = 16 \Mod{25}\\
13^{17} &= 13^{16} \cdot 13 = 16 \cdot 13 = 8 \Mod{25} \\
13^{18} &= 13^{17} \cdot 13 = 8 \cdot 13 = 4 \Mod{25} \\
13^{19} &= 13^{18} \cdot 13 = 4 \cdot 13 = 2 \Mod{25} \\
13^{20} &= 13^{19} \cdot 13 = 2 \cdot 13 = 1 \Mod{25}
\end{align*}
This shows that $\langle 13 \rangle$ generates $\mathbb{Z}^*_{25}$. To show the other generators of $\mathbb{Z}^*_{25}$, I will use corollary 5.1.8. Since $\mathbb{Z}^*_{25} = \langle 13 \rangle$ by corollary 5.1.8, $\langle 13^k \rangle$ where gcd($k,n$) = 1 is also a generator of $\mathbb{Z}^*_{25}$. So, I need to find the values of $k$ where $\gcd(20,k) = 1$, or the relative prime numbers to 20. Those are, \{1,3,7,9,11,13,17,19\}. So the generators are:
$$\langle 13 \rangle, \langle 22 \rangle, \langle 17 \rangle, \langle 23 \rangle, \langle 12 \rangle, \langle 3 \rangle, \langle 8 \rangle, \langle 2 \rangle$$
\end{problem}

\begin{problem}{11}
Without actually computing the orders, explain why the two elements 2 and 28 must have the same order in $\mathbb{Z}_{30}$. How about 8 and 22 in $\mathbb{Z}_{20}$? Do they have the same order? Do the same for the pair 2 and 8 in $\mathbb{Z}^*_{15}$ \\ \\
Using corollary 5.1.6, If $G = \langle a \rangle, \ |G| = n$ and $k|n$, then $|a^k| = n/\gcd(n,k)$, and using the fact that for all integers $n > 1, \ \mathbb{Z}_n = \langle 1 \rangle$, I will show that $|1^2| = |1^{28}|$ in $\mathbb{Z}_{30}$. Rewriting, I get $|1^2| = 30/\gcd(30,2) = 30/2 = 15$ and $|1^{28}| = 30/\gcd(30,28) = 30/2 = 15$. Therefore the order of 2 and 28 are equal. Using a similar approach I now want to show that $|1^8| = |1^{22}|$. Rewriting I get $|1^8| = 20/\gcd(20,8) =20/4 = 5$ and $|1^{22}| = 20/\gcd(20,22) = 20/2 = 10$. Therefore 8 and 22 do not have the same order. Next, I want to show that $|a^2| = |a^8|$ in $\mathbb{Z}^*_{15}$. Since the order of $\mathbb{Z}^*_{15} = 8$, and $\gcd(8,2) = 2$ and $\gcd(8,8) = 8$. This shows that $|a^2| \neq |a^8|$.
\end{problem}
\begin{problem}{12}
What are the possible orders of the elements of $D_{10}$? How many elements are there of each order?
\\ \\
The order of $D_{10}$ is 20. But $D_{10}$ is a dihedral group, so its generated by two elements. In other words, it is made up of two cyclic groups. The first cyclic group which is the rotations of $D_{10}$ is of order 10. So using theorem 5.2.2 and the divisiors of 10, 1,2,5, and 10, I find that $\phi(1) = 1, \phi(2) = 1, \phi(5) = 4, \ \text{and} \ \phi(10) = 4$. In the other cyclic group, all elements have order 2, and there are 10 of them. Since $\phi(2) = 1$, this means there are 10 elements of order 2 in this group. So, in total in $D_{10}$ there is 1 element of order 1, 11 elements of order 2, 4 elements of order 5, and 4 elements of order 10.
\end{problem}
\end{document}
