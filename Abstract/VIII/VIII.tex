%!TeX program=pdflatex
\documentclass[titlepage]{article}
 
\usepackage[margin=1in]{geometry} 
\usepackage{amsmath,amsthm,amssymb,fancyhdr,gensymb,arydshln,graphicx,setspace,mathtools}
\graphicspath{{C:/Users/user/Desktop/}}

\allowdisplaybreaks
\pagestyle{fancy}
\newcommand{\Mod}[1]{\ (\mathrm{mod}\ #1)}
\newenvironment{theorem}[2][Theorem]{\begin{trivlist}
\item[\hskip \labelsep {\bfseries #1}\hskip \labelsep {\bfseries #2.}]}{\end{trivlist}}
\newenvironment{lemma}[2][Lemma]{\begin{trivlist}
\item[\hskip \labelsep {\bfseries #1}\hskip \labelsep {\bfseries #2.}]}{\end{trivlist}}
\newenvironment{exercise}[2][Exercise]{\begin{trivlist}
\item[\hskip \labelsep {\bfseries #1}\hskip \labelsep {\bfseries #2.}]}{\end{trivlist}}
\newenvironment{problem}[2][Problem]{\begin{trivlist}
\item[\hskip \labelsep {\bfseries #1}\hskip \labelsep {\bfseries #2.}]}{\end{trivlist}}
\newenvironment{question}[2][Question]{\begin{trivlist}
\item[\hskip \labelsep {\bfseries #1}\hskip \labelsep {\bfseries #2.}]}{\end{trivlist}}
\newenvironment{corollary}[2][Corollary]{\begin{trivlist}
\item[\hskip \labelsep {\bfseries #1}\hskip \labelsep {\bfseries #2.}]}{\end{trivlist}}


\begin{document}

% --------------------------------------------------------------
%                         Start here
% --------------------------------------------------------------
 
%\title{Weekly Homework II}%replace X with the appropriate number
%\author{Dakota Wicker\\ %replace with your name
%Abstract Algebra I} %if necessary, replace with your course title
%\maketitle
%\clearpage
\fancyhf{}
\fancyhead[RO,RE]{Abstract I}
\fancyhead[LO,LE]{Dakota Wicker}
\fancyhead[CO,CE]{Homework VIII}
\cfoot{\thepage}

\begin{problem}{1}
Does there exist any cycle $\sigma$ such that $\sigma^2$ is not a cycle?
\\ \\
Yes. $(1,2,3,4)^{2} = (1,3)(2,4)$ which is not a cycle.
\end{problem}

\begin{problem}{2}
Write the following product of cycles as a cycle:
$$(2,3,6,8)(4,9,1,3)(1,5,7).$$
\\ \\
The product (4,9,1,3)(1,5,7) is equal to (1,5,7,3,4,9) and the product of  (2,3,6,8)(1,5,7,3,4,9) is equal to (1,5,7,6,8,2,3,4,9). Therefore, (2,3,6,8)(4,9,1,3)(1,5,7) = (1,5,7,6,8,2,3,4,9)
\end{problem}

\begin{problem}{3}
Determine whether 
$$\alpha = \left(\begin{array}{ccccccccc} 1 & 2 & 3 & 4 & 5 & 6 & 7 & 8 & 9 \\ 3 & 5 & 1 & 6 & 8 & 9 & 2 & 7 & 4 \end{array}\right)$$
is even or odd. How about (2,3)$\alpha$ and (3,6,1,7)$\alpha$?
\\ \\
$\alpha$ can be broken into 6 transpositions. Those are (1,3)(2,7)(2,8)(2,5)(4,9)(4,6). Since $\alpha$ requires an even number of transpositions in its decomposition $\alpha$ is even. Since $\alpha$ is even, composing another transposition would make it odd. Therefore (2,3)$\alpha$ is odd. (3,6,1,7)$\alpha$ is also odd because (3,6,1,7) can be composed into 3 transpositions, (3,7)(3,1)(3,6) and an even number of transpositions plus an odd number of transpositions is always odd so (3,6,1,7)$\alpha$ is odd.
\end{problem}

\begin{problem}{4}
Determine whether the following statement is true or false, and explain:
$$ \gamma = \left(\begin{array}{ccccccccc} 1 & 2 & 3 & 4 & 5 & 6 & 7 & 8  \\ 2 & 6 & 7 & 5 & 8 & 1 & 3 & 4 \end{array}\right) \in A_8.$$
\\ 
$\gamma \not\in A_8$ because $\gamma$ is represented by an odd number of transpositions. Those are (1,6)(1,2)(3,7)(4,8)(4,5). Since $A_8$ contains only even permutations, $\gamma$ cannot be in $A_8$. So, the statement is false.
\end{problem}

\begin{problem}{5}
Consider 
$$ \beta = \left(\begin{array}{ccccccccc} 1 & 2 & 3 & 4 & 5 & 6 & 7 & 8  \\ 2 & 8 & 5 & 6 & 4 & 7 & 3 & 1\end{array}\right) \in S_8.$$ 
\begin{itemize}
\item[(a)] Evaluate $\beta^{2014}$.
\item[(b)] Find $\alpha \in S_8$ such that $\alpha^4 = \beta$.
\end{itemize}
\hrule
\begin{itemize}
\item[(a)] Since $\beta$ can be written as $\beta = $ (1,2,8)(3,5,4,6,7), I know that the order of $\beta$ is equal to the LCM of the lengths of these cycles. That is, ord($\beta$) = LCM(3,5) = 15. Since the order is 15, that means that $\beta^{2010} = \iota$ because 2010 is divisible by 15. $\beta^{2014} = \beta^{2010}\beta^4$, so $\beta^{2014} = \iota \beta^4 = \beta^4 = \left(\begin{smallmatrix} 1 & 2 & 3 & 4 & 5 & 6 & 7 & 8 \\ 2 & 8 & 7 & 5 & 3 & 4 & 6 & 1 \end{smallmatrix}\right)$.

\item[(b)] Since $\beta$ is written as (1,2,8)(3,5,4,6,7), it is written as two cycles with different lengths. This means that $\alpha$ must be the product of at least two cycles of different lengths. Assume $\alpha = \gamma\sigma$, so that $\alpha^4 = \gamma^4\sigma^4 = (1,2,8)(3,5,4,6,7)$. $\gamma$ is a 3-cycle because by theorem 6.4.1 if $\gamma$ is an m-cycle, then $\gamma^4$ is a product of $\gcd(m,4)$ cycles and has length $\frac{m}{\gcd(m,k)}$. Since the length of $\gamma^4$ is 3, $\frac{m}{\gcd(m,4)} = 3$. Since $\gamma^4$ is a product of 1 cycle, $\gcd(m,4) = 1$. Both of these equations are solved when $m=3$. Therefore $\gamma$ is a 3-cycle. Using this fact, $\gamma^4 = \gamma^3 \gamma = \iota \gamma = \gamma$. $\sigma$ is a 5-cycle because by theorem 6.4.1, if $\sigma$ is an m-cycle, then $\sigma^4$ has length $\frac{m}{\gcd(m,4)}$ and is a product of $\gcd(m,4)$ cycles. Since the length of $\sigma^4 = 5$, $\frac{m}{\gcd(m,4)} = 5$. Since $\sigma^4$ is a product of 1 cycle, $\gcd(m,4) = 1$. Both equations are satasfied when $m=5$. Therefore $\sigma$ is a 5-cycle. Knowing that $\sigma$ is a 5-cycle, it is easy to see that $\sigma = (3,7,6,4,5)$. So, $\alpha = \gamma\sigma =(1,2,8)(3,7,6,4,5)$.
\end{itemize}
\end{problem}

\begin{problem}{6}
For any $H\leq S_n$ and any fixed permutation $\sigma \in S_n$, define 
$$K = \left\{\sigma\tau\sigma^{-1} \ | \ \tau \in H \right\}.$$
In other words, $K$ is the set consisting of all permutations of the form $\sigma\tau\sigma^{-1}$ for some element $\tau \in H$. Show that $K \leq S_n$. 
\\ \\
Since $H$ is a subgroup of $S_n$ it contains $\iota$. If I let $\sigma = \iota$ and $\tau = \iota$, this shows that $\iota \in K$ because $\iota\iota\iota = \iota$. So, $K$ is a non-empty subset of $S_n$. $K$ is closed under $S_n$'s binary operation. To show this I will show the product of two elements $a,b \in K$ is also in $K$. Since $a,b \in K$ $a =\sigma \tau_1 \sigma^{-1}, b = \sigma \tau_2 \sigma^{-1}$. So the product is $ab = \sigma \tau_1 \sigma^{-1}\sigma \tau_2 \sigma^{-1} = \sigma \tau_1 \iota \tau_2 \sigma^{-1} = \sigma \tau_1 \tau_2 \sigma^{-1}$. Since $\tau \in H$ and $H \leq S_n$, $H$ is closed so, let the product $\tau_1 \tau_2 = \tau_3$. Then, $ab = \sigma \tau_3 \sigma^{-1}$. This takes the form of $K$ so $ab \in K$ therefore $K$ is closed. For all $a \in K$, the inverse also exists. That is, there is an element $a^{-1} \in K$ such that $aa^{-1} = \iota$. Rewritten, this is $\sigma \tau_1 \sigma^{-1} \sigma \tau_2 \sigma = \sigma \tau_1 \tau_2 \sigma^{-1} = \iota$. Since I want to show that $aa^{-1} = \iota$, it must be that $\tau_2 = \tau_1^{-1}$ which exists because $\tau_1 \in H$ and $H \leq S_n$. So, $aa^{-1} = \sigma\tau_1\tau_1^{-1}\sigma^{-1} = \sigma\sigma^{-1} = \iota$. This shows that the inverse exists for all $a \in K$. Since I have shown $K$ is non-empty, the binary operation of $S_n$ is closed in $K$, and for all $a \in K$ $a^{-1}$ exists, by corollary 4.2.2 $K \leq S_n$. 

\end{problem}

\begin{problem}{7}
Let $\alpha$ be a 10-cycle. Find the integers $k$, where $2 \leq k \leq 10$, such that $\alpha^k$ also a 10-cycle? Explain.	
\\ \\
To find the values of $k$ where $\alpha^k$ is a 10-cycle, I will use theorem 6.4.1. According to the theorem, $\alpha^k$ is a product of $\gcd(10,k)$ cycles and has length $\frac{10}{\gcd(10,k)}$. Since I want to find the values of $k$ such that $\alpha^k$ is a 10-cycle, then $\frac{10}{\gcd(10,k)} = 10.$ The values in which this happens is when $k=1,3,7,9$. But, I also want the number of cycle products to be 1 and the k value to not be greater than one and less than 11. The $k$ values that satisfy this are $k = 3,7,9$. 
\end{problem}

\begin{problem}{8}
Let $\beta \in S_n$ such that $\text{ord}_{S_n}(\beta) = 36$.
\begin{itemize}
\item[(a)] How many elements are there in $\langle \beta \rangle$?
\item[(b)] What are the generators of $\langle \beta \rangle$?
\item[(c)] Which subgroup $H$ of $\langle \beta \rangle$ has order 12?
\item[(d)] Find the generators of the subgroup $H$ of $\langle \beta \rangle$ of order 12.
\item[(e)] How many elements in $\langle \beta \rangle$ have order 12? Does your answer agree with part (d)? Explain.
\end{itemize}
\hrule
\begin{itemize}
\item[(a)] Since the order of $\beta$ is 36, this means that there are 36 elements in $\langle \beta \rangle$.
\item[(b)] Using theorem 5.1.6 $|\beta^k| = \frac{36}{\gcd(36,k)}$. Since I want to find the generators of $\beta$, I need the orders to be the same. That is, $|\beta^k| = \frac{36}{\gcd(36,k)} = 36$. This is when $\gcd(36,k) = 1$. Solving for $k$ I find that they are all that are coprime to 36. $k=1,5,7,11,13,17,19,23,25,29,31,35$. But the generators of $\beta$ must be $\{\beta, \beta^5, \beta^7, \beta^{11}, \beta^{13}, \beta^{17}, \beta^{19}, \beta^{23}, \beta^{25}, \beta^{29}, \beta^{31}, \beta^{35}\}$.
\item[(c)] $H = \{ \beta^{3n} \ | \ n \in \mathbb{Z}_{12}\}$
\item[(d)] Again using theorem 5.1.6 I want to find the values of $k$ where $|\beta^k| = \frac{36}{\gcd(36,k)} = 12.$ This is when $k = 3,15,21,33$. So, the generators of $H$ are $\{\beta^3, \beta^{15}, \beta^{21}, \beta^{33} \}$.
\item[(e)] There are 4 elements in $\langle \beta \rangle$ which I computed in (d). Those are $\beta^k$ when $k=3,15,21,33$. By using the theorem 5.2.2 and the euler totient function, I should get $\phi(12) = 4$ values. Since I got 4 values and was expecting 4 values my answer agrees with part (d).
\end{itemize}
\end{problem}

\begin{problem}{9}
It can be shown that (but you do not have to)
$$H = \left\{\left(\begin{array}{cccc} 1 & 2 & 3 & 4 \\ 1 & 2 & 3 & 4\end{array} \right), \left(\begin{array}{cccc} 1 & 2 & 3 & 4 \\ 2 & 3 & 4 & 1\end{array}  \right), \left(\begin{array}{cccc} 1 & 2 & 3 & 4 \\ 3 & 4 & 1 & 2\end{array} \right), \left(\begin{array}{cccc} 1 & 2 & 3 & 4 \\ 4 & 1 & 2 & 3 \end{array} \right)\right\} $$
is a group.
\begin{itemize}
\item[(a)] Compute the operation table for $H$.
\item[(b)] For brevity, name the four permutations as $\iota,\alpha,\beta,\gamma.$ Rewrite the operation table in terms of $\iota,\alpha,\beta,\gamma.$
\item[(c)] This operation table is identical to the operation table of which familiar group?
\item[(d)] Express $\beta$ and $\gamma$ in terms of $\alpha$.
\item[(e)] What is the order of $\alpha$? Does this conform with your answer to part (c)?
\item[(f)] What are the generators of $H$?
\end{itemize}
\hrule

\begin{itemize}
\item[(a)]  \begin{tabular}{c | c c c c}
    $\cdot$ & $\left(\begin{smallmatrix} 1 & 2 & 3 & 4 \\ 1 & 2 & 3 & 4\end{smallmatrix} \right)$ & $\left(\begin{smallmatrix} 1 & 2 & 3 & 4 \\ 2 & 3 & 4 & 1\end{smallmatrix}\right)$ & $\left(\begin{smallmatrix} 1 & 2 & 3 & 4 \\ 3 & 4 & 1 & 2\end{smallmatrix}\right)$ & $\left(\begin{smallmatrix} 1 & 2 & 3 & 4 \\ 4 & 1 & 2 & 3 \end{smallmatrix} \right)$ \\
    \cline{1-5}
    $\left(\begin{smallmatrix} 1 & 2 & 3 & 4 \\ 1 & 2 & 3 & 4\end{smallmatrix} \right)$ & $\left(\begin{smallmatrix} 1 & 2 & 3 & 4 \\ 1 & 2 & 3 & 4\end{smallmatrix} \right)$ & $\left(\begin{smallmatrix} 1 & 2 & 3 & 4 \\ 2 & 3 & 4 & 1\end{smallmatrix}\right)$ & $\left(\begin{smallmatrix} 1 & 2 & 3 & 4 \\ 3 & 4 & 1 & 2\end{smallmatrix}\right)$ & $\left(\begin{smallmatrix} 1 & 2 & 3 & 4 \\ 4 & 1 & 2 & 3 \end{smallmatrix} \right)$ \\
    $\left(\begin{smallmatrix} 1 & 2 & 3 & 4 \\ 2 & 3 & 4 & 1\end{smallmatrix}\right)$ & $\left(\begin{smallmatrix} 1 & 2 & 3 & 4 \\ 2 & 3 & 4 & 1\end{smallmatrix}\right)$ & $\left(\begin{smallmatrix} 1 & 2 & 3 & 4 \\ 3 & 4 & 1 & 2\end{smallmatrix}\right)$ &  $\left(\begin{smallmatrix} 1 & 2 & 3 & 4 \\ 4 & 1 & 2 & 3 \end{smallmatrix} \right)$ & $\left(\begin{smallmatrix} 1 & 2 & 3 & 4 \\ 1 & 2 & 3 & 4\end{smallmatrix} \right)$ \\
    $\left(\begin{smallmatrix} 1 & 2 & 3 & 4 \\ 3 & 4 & 1 & 2\end{smallmatrix}\right)$& $\left(\begin{smallmatrix} 1 & 2 & 3 & 4 \\ 3 & 4 & 1 & 2\end{smallmatrix}\right)$ & $\left(\begin{smallmatrix} 1 & 2 & 3 & 4 \\ 4 & 1 & 2 & 3 \end{smallmatrix} \right)$ & $\left(\begin{smallmatrix} 1 & 2 & 3 & 4 \\ 1 & 2 & 3 & 4\end{smallmatrix} \right)$ &$\left(\begin{smallmatrix} 1 & 2 & 3 & 4 \\ 2 & 3 & 4 & 1\end{smallmatrix}\right)$\\
    $\left(\begin{smallmatrix} 1 & 2 & 3 & 4 \\ 4 & 1 & 2 & 3 \end{smallmatrix} \right)$ & $\left(\begin{smallmatrix} 1 & 2 & 3 & 4 \\ 4 & 1 & 2 & 3 \end{smallmatrix} \right)$ & $\left(\begin{smallmatrix} 1 & 2 & 3 & 4 \\ 1 & 2 & 3 & 4\end{smallmatrix} \right)$ & $\left(\begin{smallmatrix} 1 & 2 & 3 & 4 \\ 2 & 3 & 4 & 1\end{smallmatrix}\right)$ & $\left(\begin{smallmatrix} 1 & 2 & 3 & 4 \\ 3 & 4 & 1 & 2\end{smallmatrix}\right)$ \\
\end{tabular}

\item[(b)]  \begin{tabular}{c | c c c c}
    $\cdot$ & $\iota$ & $\alpha$ & $\beta$ & $\gamma$ \\
    \cline{1-5}
    $\iota$ & $\iota$ & $\alpha$ & $\beta$ & $\gamma$ \\
    $\alpha$ & $\alpha$ & $\beta$ &  $\gamma$ & $\iota$ \\
    $\beta$& $\beta$ & $\gamma$ & $\iota$ &$\alpha$\\
    $\gamma$ & $\gamma$ & $\iota$ & $\alpha$ & $\beta$ \\
\end{tabular}
\item[(c)] The subgroup of rotations in $D_4$
\item[(d)] $\beta = \alpha^2$ and $\gamma = \alpha\cdot\beta = \alpha^3$
\item[(e)] ord($\alpha$) = 4. This does conform with my answer to part (c).
\item[(f)] The generators of $H$ are $\alpha$ and $\gamma$ because if I multiply them by themselves I get all elements in $H$ and reach $\iota$ when the power is 4.
\end{itemize}
\end{problem}
\begin{problem}{10}
Determine whether the function $\phi: \langle \mathbb{R},+ \rangle \rightarrow \langle \mathbb{R}^+,\cdot\rangle$ defined by $\phi(x)=3^{x/2}$ is an isomorphism between the two groups $\langle \mathbb{R},+ \rangle$ and $\langle \mathbb{R}^+,\cdot\rangle$
\\ \\
To determine that $\phi$ is an isomorphism between $\langle \mathbb{R},+ \rangle$ and $\langle \mathbb{R}^+,\cdot\rangle$, I will need to show that there is a bijection between these two and that $\phi$ is operation preserving. $\phi$ forms a bijection between $\langle \mathbb{R},+ \rangle$ and $\langle \mathbb{R}^+,\cdot\rangle$. To show this I show that $\phi$ is onto. That is, $\phi(x_1) = \phi(x_2) \implies x_1=x_2$. Assume $\phi(x_1) = \phi(x_2)$. Then, I can rewrite this as $3^{x_1/2} = 3^{x_2/2}$. If I take the natural log of both sides I get $\frac{x_1}{2}\ln(3) = \frac{x_2}{2}\ln(3)$. Multiplying by two and then dividing by $\ln(3)$ on both sides gives me $x_1 = x_2$. Next it is easy to see that $\phi$ is onto because for every element $y \in \mathbb{R}^+$ there is an element $x \in \mathbb{R}$ such that $\phi(x) = y$. Since $\phi$ is one to one and onto, $\phi$ is a bijection. $\phi$ is also operation preserving. This is true because $\phi(x_1+x_2) = \phi(x_1)\cdot\phi(x_2)$. Looking at the left hand side, $\phi(x_1+x_2) = 3^{\frac{x_1+x_2}{2}}$. Now looking at the right hand side, $\phi(x_1) \cdot \phi(x_2) = 3^{\frac{x_1}{2}}\cdot 3^{\frac{x_2}{2}} = 3^{\frac{x_1}{2} + \frac{x_2}{2}} = 3^{\frac{x_1 + x_2}{2}}$. Since $\phi(x_1+x_2) = \phi(x_1)\cdot\phi(x_2) = 3^{\frac{x_1 + x_2}{2}}$, $\phi$ is operation preserving. Finally, since $\phi$ forms a bijection between $\langle \mathbb{R},+ \rangle$ and $\langle \mathbb{R}^+,\cdot\rangle$ and is operating preserving, $\phi$ is an isomorphism between $\langle \mathbb{R},+ \rangle$ and $\langle \mathbb{R}^+,\cdot\rangle$.
\end{problem}

\begin{problem}{11}
Show that $\langle 5\mathbb{Z}, + \rangle \cong \langle 8\mathbb{Z}, + \rangle$
\\ \\
To show that $\langle 5\mathbb{Z}, + \rangle \cong \langle 8\mathbb{Z}, + \rangle$, I need to show that there is a function $\phi:\langle 5\mathbb{Z}, + \rangle \rightarrow \langle 8\mathbb{Z}, + \rangle$ that is a bijection and is operation preserving. Noticing a common pattern I guess the function $\phi(x) = x + \frac{3x}{5}$ forms an isomorphism. To show that it does, I first show a bijection. Assume $\phi(x_1)=\phi(x_2)$, this implies $x_1 = x_2.$ I show this by finding that $\phi(x_1) = x_1 + \frac{3x_1}{5} = x_1(1+\frac{3}{5})$ and $\phi(x_2) = x_2 + \frac{3x_2}{5} = (x_2)(1+\frac{3}{5})$. Since I assume $\phi(x_1)=\phi(x_2)$, then $x_1(1+\frac{3}{5}) = x_2(1+\frac{3}{5})$. Since $(1+\frac{3}{5})$ is constant it is clear that $x_1 = x_2$. This shows that $\phi$ is one to one. To show that $\phi(x)$ is onto, I must show that for every $y \in 8\mathbb{Z}$, there is an $ x \in 5\mathbb{Z}$ such that $\phi(x) = y$. I do this by rewriting $\phi$ as $\phi(x) = x(1+\frac{3}{5}) = \frac{8x}{5} = 8(\frac{x}{5})$. Clearly, since $x\in 5\mathbb{Z}, \frac{x}{5}$ is an integer and $8(\frac{x}{5}) \in 8\mathbb{Z}$ when $\frac{x}{5}$ is an integer. This shows that $\phi$ is onto. $\phi$ is also operation preserving. To show this I must show that $\phi(x_1 + x_2) = \phi(x_1) + \phi(x_2)$. Expanding the left hand side I get $\phi(x_1 + x_2) = (x_1+x_2)(1 + \frac{3}{5})$. Expanding the right hand side I get $\phi(x_1) + \phi(x_2) = (x_1)(1 + \frac{3}{5}) + (x_2)(1+\frac{3}{5}) = (1+\frac{3}{5})(x_1 + x_2)$. Since $\phi(x_1 + x_2) = (x_1 + x_2)(1 + \frac{3}{5}) = \phi(x_1) + \phi(x_2)$, this shows that $\phi$ is operation preserving. Finally, since $\phi$ is a bijection that is also operation preserving, $\phi$ is an isomorphism between $\langle 5\mathbb{Z},+ \rangle$ and $\langle 8\mathbb{Z} \rangle$. Since an isomorphism exists between these two groups, $\langle 5\mathbb{Z}, + \rangle \cong \langle 8\mathbb{Z}, + \rangle$.
\end{problem}

\begin{problem}{12}
Let 
$$G = \left\{a + b\sqrt{2} \ \big| \ a,b\in \mathbb{Q}\right\} $$
and
$$ H = \left\{ \begin{bmatrix} x & 2y \\ y & x\end{bmatrix} \ \bigg| \ x,y \in \mathbb{Q} \right\}$$
It is known that $\langle G,+ \rangle$ and $\langle H, + \rangle$ are groups. Show that $\langle G,+ \rangle \cong \langle H,+ \rangle$.
\\ \\
To show that  $\langle G,+ \rangle \cong \langle H,+ \rangle$ I will find a function, $\phi$, that is an isomorphism between these two groups. The function $\phi:\langle G,+ \rangle \rightarrow \langle H,+ \rangle $ I define as $\phi(c) = \left[\begin{smallmatrix} a & 2b \\ b & a\end{smallmatrix}\right]$ is an isomorphism. To show this I will show that there is a bijection and that $\phi$ is operation preserving. Assume $\phi(c_1) = \phi(c_2).$ Expanding the left side of this equation I get, $\phi(c_1) = \left[\begin{smallmatrix}a_1 & 2b_1 \\ b_1 & a_1\end{smallmatrix}\right]$. Expanding the right side I get $\phi(c_2) = \left[\begin{smallmatrix}a_2 & 2b_2 \\ b_2 & a_2\end{smallmatrix}\right]$. This shows that $\left[\begin{smallmatrix}a_1 & 2b_1 \\ b_1 & a_1\end{smallmatrix}\right] =\left[\begin{smallmatrix}a_2 & 2b_2 \\ b_2 & a_2\end{smallmatrix}\right]$. Since the matricies are equal, the entries must be equal such that $a_1 = a_2, 2b_1 = 2b_2$, and $b_1 = b_2$. Therefore, $c_1 = a_1 + b_1\sqrt{2} = a_2 + b_2\sqrt{2} = c_2$. This shows that $\phi$ is one to one. $\phi$ is onto because for every $d \in H$, there is clearly an $e \in G$ such that $\phi(e) = d$. $\phi$ is operation preserving. This is because $\phi(c_1 + c_2) = \phi(c_1) + \phi(c_2)$. I will show this by expanding the left hand side to be $\phi(c_1 + c_2) = \phi(a_1+b_1\sqrt{2} + a_2 +b_2\sqrt{2}) =\left[\begin{smallmatrix}a_1+a_2 & 2(b_1+b_2) \\ b_1+b_2 & a_1+a_2\end{smallmatrix}\right]$. Now I will expand the right hand side of the equation to get $\phi(c_1) + \phi(c_2) = \left[\begin{smallmatrix}a_1 & 2b_1 \\ b_1 & a_1\end{smallmatrix}\right] + \left[\begin{smallmatrix}a_2 & 2b_2 \\ b_2 & a_2\end{smallmatrix}\right] = \left[\begin{smallmatrix}a_1 +a_2  & 2(b_1+b_2) \\ b_1+b_2 & a_1+a_2\end{smallmatrix}\right]$. This shows that $\phi(c_1+c_2) = \left[\begin{smallmatrix}a_1 +a_2  & 2(b_1+b_2) \\ b_1+b_2 & a_1+a_2\end{smallmatrix}\right] = \phi(c_1) + \phi(c_2)$ and therefore $\phi$ is operation preserving. Since $\phi$ is a bijection and operation preserving function between $\langle G,+ \rangle $ and $ \langle H,+ \rangle$, $\phi$ is an isomorphism between $\langle G,+ \rangle $ and $\langle H,+ \rangle$. Since an isomorphism exists between these two groups, $\langle G,+ \rangle \cong \langle H,+ \rangle$.
\end{problem}
\end{document}