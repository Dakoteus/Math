%!TeX program = lualatex
\documentclass[titlepage]{article}
\usepackage{../Head}
\usepackage{relsize}
\graphicspath{.}
\begin{document}
\fancyhf{}
\fancyhead[RO,R]{Advanced Calculus 420}
\fancyhead[LO,L]{Dakota Wicker}
\fancyhead[CO,C]{Homework X}
\cfoot{\thepage}

\begin{cproblem}{1}{White}
Utilizing the parameterization of $x,y,z$ in terms of $u,v$, prove the following rules of the product of differential forms.
\begin{itemize}
\item[1.] $d(x+y)dz = dxdz + dydz$
\item[2.] $dxdy = -dydx$. In particular, $dxdx=0$
\end{itemize}
\end{cproblem}
\begin{solution}
\vspace{-2em}
\begin{itemize}
\item[1.]
\begin{proof}
 From the definitions of $dxdz$ and $dydz$ I expand out the R.H.S of the equation to be 
$$dx\,dz + dy\,dz = \frac{\p(x,z)}{\p(u,v)}dudv  + \frac{\p (y,z)}{\p (u,v)}dudv.$$
It is also true that 
\begin{align*}
\frac{\p (x,z)}{\p (u,v)}dudv &= det\left(\bmat{\frac{\p x}{\p u} & \frac{\p x}{\p v} \\ \frac{\p z}{\p u} & \frac{\p z}{ \p v}}\right)dudv, \\
\frac{\p (y,z)}{\p (u,v)}dudv & =det\left(\bmat{\frac{\p y}{\p u} & \frac{\p y}{\p v} \\ \frac{\p z}{\p u} & \frac{\p z}{ \p v}}\right)dudv .
\end{align*}
Using the fact that $det[A+B|C] = det[A|C]+det[B|C]$ I rewrite the R.H.S of the original equation to be
\begin{align*}dx\,dz + dy\,dz &=\left( det\left(\bmat{\frac{\p x}{\p u} & \frac{\p x}{\p v} \\ \frac{\p z}{\p u} & \frac{\p z}{ \p v}}\right) + det\left(\bmat{\frac{\p y}{\p u} & \frac{\p y}{\p v} \\ \frac{\p z}{\p u} & \frac{\p z}{ \p v}}\right)\right)dudv\\
 &= det\left(\bmat{\frac{\p x}{\p u} + \frac{\p y}{\p u} & \frac{\p x}{\p v} + \frac{\p y}{\p v} \\ \frac{\p z}{\p u} & \frac{\p z}{ \p v}}\right)dudv\\
  &= \frac{\p(x+y,z)}{\p (u,v)}dudv.
\end{align*}
Now, notice that 
$$d(x+y)dz = \frac{\p (x + y,z)}{\p (u,v)}du\,dv$$
By substitution,
$$d(x+y)dz = \frac{\p (x + y,z)}{\p (u,v)}du\,dv =  dx\,dz + dy\,dz$$
Therefore, $d(x+y)dz = dxdz + dydz$.
\end{proof}
\item[2.]
\begin{proof} 
I will rewrite the L.H.S of the original equation to be
$$ dxdy = \frac{\p(x,y)}{\p(u,v)}dudv = det\left( \bmat{\frac{\p x}{\p u} & \frac{\p x}{\p v} \\ \frac{\p y}{\p u} & \frac{\p y}{\p v}}\right)dudv $$
and I can also write
$$ dydx =  \frac{\p(y,x)}{\p(u,v)}dudv =det\left( \bmat{\frac{\p y}{\p u} & \frac{\p y}{\p v} \\ \frac{\p x}{\p u} & \frac{\p x}{\p v}}\right)dudv $$
which implies that 
$$-dydx = - det\left( \bmat{\frac{\p y}{\p u} & \frac{\p y}{\p v} \\ \frac{\p x}{\p u} & \frac{\p x}{\p v}}\right)dudv.$$
Using the property that $det([A|B]) = -det([B|A])$, it follows that
$$ det\left( \bmat{\frac{\p x}{\p u} & \frac{\p x}{\p v} \\ \frac{\p y}{\p u} & \frac{\p y}{\p v}}\right)dudv = - det\left( \bmat{\frac{\p y}{\p u} & \frac{\p y}{\p v} \\ \frac{\p x}{\p u} & \frac{\p x}{\p v}}\right)dudv $$
which then by substitution on the previous equation means that
$$dxdy = -dydx.$$
\end{proof}
\end{itemize}
\end{solution}

\begin{cproblem}{2}{White}\ \\\
\vspace{-1em}
\begin{itemize}
\item[a.] Use the rules for multiplying differential forms on cylindrical coordinates
$$ x = r\cos(\theta), \ y = r\sin(\theta), \ z = z $$
to show that
$$ dxdydz = r\,drd\theta dz.$$
\item[b.] Use the rules for multiplying differential forms on spherical coordinates
$$ x = \rho\sin(\phi)\cos(\theta), \ y = \rho\sin(\phi)\sin(\theta), \ z = \rho\cos(\phi) $$
to show that
$$ dxdydz = \rho^2\sin(\phi)d\rho d\phi d\theta.$$
\end{itemize}
\end{cproblem}
\begin{solution}
\vspace{-2em}
\begin{itemize}
\item[a.] It is true that $dxdydz = (dxdy)dz$. Expanding this using the multiplication rules for differential forms I get
\begin{align*}dxdydz=(dxdy)dz &=  \left(\frac{\p (x,y)}{\p (r,\theta)}drd\theta\right)dz = \left(det\left(\bmat{\cos(\theta) & -r\sin(\theta) \\ \sin(\theta) & r\cos(\theta)}\right)drd\theta\right)dz\\
&= r(\cos^2(\theta) + \sin^2(\theta))drd\theta dz = r\,drd\theta dz
\end{align*}
This shows that $dxdydz = r\,drd\theta dz.$
\item[b.] By expanding out the L.H.S of the equation I get
$$ dxdydz = \left( \frac{\p x}{\p \rho} d\rho + \frac{\p x}{\p \phi} d\phi + \frac{\p x}{\p \theta} d\theta\right)\left( \frac{\p y}{\p \rho} d\rho + \frac{\p y}{\p \phi} d\phi + \frac{\p y}{\p \theta} d\theta\right) \left( \frac{\p z}{\p \rho} d\rho + \frac{\p z}{\p \phi} d\phi + \frac{\p z}{\p \theta} d\theta\right).$$
Multiplying the first two terms out I get,
\begin{align*}
dxdy =& \sin^2(\phi)\cos(\theta)\sin(\theta)d\rho d\rho + \rho\sin(\phi)\cos(\phi)\cos(\theta)\sin(\theta) d\rho d\phi \\
&\,\hspace{-.1em}+ \rho\sin^2(\phi)\cos^2(\theta) d\rho d\theta + \rho \cos(\phi)\sin(\phi)\cos(\theta)\sin(\theta)d\phi d\rho \\
&\, \hspace{-.1em}+ \rho^2 \cos^2(\phi)\cos(\theta)\sin(\theta)d\phi d\phi + \rho^2 \cos(\phi)\sin(\phi)\cos^2(\theta) d\phi d\theta \\
&\, \hspace{-.1em}- \rho\sin^2(\phi)\sin^2(\theta)d\theta d\rho - \rho^2\sin(\phi)\cos(\phi)\sin^2(\theta) d\theta d\phi \\
&\, \hspace{-.1em}- \rho^2\sin^2(\phi)\sin(\theta)\cos(\theta) d\theta d\theta.
\end{align*}
I can now simplify this because of the property that a differential form multiplied by itself is zero. This gives me
\begin{align*}
dxdy =& \rho\sin(\phi)\cos(\phi)\cos(\theta)\sin(\theta) d\rho d\phi \\
&\,\hspace{-.1em}+ \rho\sin^2(\phi)\cos^2(\theta) d\rho d\theta + \rho \cos(\phi)\sin(\phi)\cos(\theta)\sin(\theta)d\phi d\rho \\
&\, \hspace{-.1em}+ \rho^2 \cos(\phi)\sin(\phi)\cos^2(\theta) d\phi d\theta \\
&\, \hspace{-.1em}- \rho\sin^2(\phi)\sin^2(\theta)d\theta d\rho - \rho^2\sin(\phi)\cos(\phi)\sin^2(\theta) d\theta d\phi.
\end{align*}
Using property 2 of problem 1 and rewriting I get that 
$$dxdy = \rho\sin^2(\phi)(\cos^2(\theta) + \sin^2(\theta))d\rho d\theta + \rho^2\cos(\phi)\sin(\phi)(\cos^2(\theta) + \sin^2(\theta))d\phi d\theta$$
and rewriting this I get
$$dxdy = \rho\sin^2(\phi)d\rho d\theta + \rho^2 \cos(\phi)\sin(\phi)d\phi d\theta.$$
Now, multiplying this by $dz$ I get
$$dxdydz = (\rho\sin^2(\phi)d\rho d\theta + \rho^2 \cos(\phi)\sin(\phi)d\phi d\theta)(\cos(\phi) d\rho - \rho \sin(\phi) d\phi)$$
which then factors out to be
$$ dxdydz = (\rho\sin^2(\phi)\cos(\phi)d\rho d\theta d\rho - \rho^2 \sin^3(\phi) d\rho d\theta d\phi + \rho^2 \cos^2(\phi)\sin(\phi) d\phi d\theta d\rho - \rho^3 \cos(\phi) \sin^2(\phi) d\phi d\theta d\phi) $$
and by using the fact that a differential multiplied by itself is zero and property 2 of problem 1, I get that
$$dxdydz= - \rho^2 \sin^3(\phi) d\rho d\theta d\phi + \rho^2 \cos^2(\phi)\sin(\phi) d\phi d\theta d\rho = \rho^2 \sin(\phi)d\rho d\phi d\theta.$$
\end{itemize}
\end{solution}
\begin{cproblem}{3}{White} For each of the following differential forms $\omega$, compute $d\omega$. Use the rules of multiplication to combine terms as much as possible in your answer.
\begin{itemize}
\item[a.] $\omega = (x^2 - z)dx + (yz)dy + (e^x + y^3)dz$
\item[b.] $\omega = (\sin(xz))dydz + (x+yz)dzdx + (xyz)dxdy$
\item[c.] $\omega = f(r) = f\left(\sqrt{x^2 + y^2}\right)$
\end{itemize}
\end{cproblem}
\begin{solution}
\vspace{-2em}
\begin{itemize}
\item[a.] By using the rules for differential forms I get 
$$d\omega = (2dxdx - dz)dx + (zdy + ydz)dy + (e^xdx + 3y^2dy)dz = -dzdx + ydzdy +  (e^xdx + 3y^2dy)dz $$ 
$$= -(e^x + 1)dzdx + (3y^2 - y)dydz$$
\item[b.]By using the rules for differential forms I get
$$ d\omega = (z\cos(x) dx + x\cos(z)dz)dydz + (dx + zdy + ydz)dzdx + (yzdx + xzdy + xydz)dxdy$$
which becomes
$$d\omega  = z\cos(x)dxdydz + zdydzdx + xydzdxdy = (z\cos(x) + z + xy)dxdydz$$
\item[c.] By using the rules for differential forms I get that
$$ d\omega = d(f(r))  = \frac{\p f}{\p r}\frac{\p r}{\p x}dx + \frac{\p f}{\p r}\frac{\p r}{\p y}dy = \frac{\p f}{\p r}\frac{1}{\sqrt{x^2 + y^2}}(xdx+ydy).$$
\end{itemize}
\end{solution}
\begin{cproblem}{4}{White}
Suppose that $\omega = f(x,y,z)dx + g(x,y,z)dy + h(x,y,z)dz$ is a 1-form defined on a 2-dimensional surface $S\subset \R^3$ and its (1-dimensional) boundary $\p S$. Show that
$$\boint{\p S}{} \omega = \bigiint{S}{} d\omega$$
which becomes
$$\boint{\p S}{} f\,dx + g\,dy + h\,dz = \bigiint{S}{} \left(\frac{\p h}{\p y} - \frac{\p g}{\p z}\right)dy\,dz + \left(\frac{\p f}{\p z} - \frac{\p h}{\p x}\right)dz\,dx + \left(\frac{\p g}{\p x} - \frac{\p f}{\p y}\right)dx\,dy$$
which is (the original) Stokes' Theorem.
\end{cproblem}
\begin{solution}
Computing $d\omega$ I get
$$d\omega = df\,dx + dg\,dy + dh\,dz$$
I rewrite this as
$$d\omega = \left(\frac{\p f}{\p x} dx + \frac{\p f}{\p y} dy + \frac{\p f}{\p z} dz\right)dx + \left(\frac{\p g}{\p x} dx + \frac{\p g}{\p y} dy + \frac{\p g}{\p z} dz\right)dy + \left(\frac{\p h}{\p x} dx + \frac{\p h}{\p y} dy + \frac{\p h}{\p z} dz\right)dz.$$
Using properties about differential forms shown in previous problems I simplify to get 
$$d\omega = \frac{\p f}{\p y} dy\,dx + \frac{\p f}{\p z} dz\,dx + \frac{\p g}{\p x} dx\,dy + \frac{\p g}{\p z} dz\,dy + \frac{\p h}{\p x} dx\,dz + \frac{\p h}{\p y} dy\,dz$$
which combines to
$$ d\omega = \left(\frac{\p h}{\p y} - \frac{\p g}{\p z}\right)dy\,dz + \left(\frac{\p f}{\p z} - \frac{\p h}{\p x}\right)dz\,dx + \left(\frac{\p g}{\p x} - \frac{\p f}{\p y}\right)dx\,dy.$$
Given that
$$\boint{\p S}{} \omega = \bigiint{S}{} d\omega$$
By substitution and since $\omega$ is a 1-form defined on a surface with a 1-dimensional boundary $\p S$ and $d\omega$ is a 2-form on $S$, it follows that 
$$\boint{\p S}{} f\,dx + g\,dy + h\,dz = \bigiint{S}{} \left(\frac{\p h}{\p y} - \frac{\p g}{\p z}\right)dy\,dz + \left(\frac{\p f}{\p z} - \frac{\p h}{\p x}\right)dz\,dx + \left(\frac{\p g}{\p x} - \frac{\p f}{\p y}\right)dx\,dy$$
which is Stokes' Theorem.
\end{solution}

\begin{cproblem}{5}{White}
Suppose that $\omega = f(x,y,z)dy\,dz + g(x,y,z)dz\,dx + h(x,y,z)dx\,dy$ is a 2-form defined on a 3-dimensional region $S \subset \R^3$ and its 2-dimensional boundary $\p S$. Show that
$$\boint{\p S}{} \omega = \bigiint{S}{} d\omega $$
becomes
$$\mathlarger\oiint\displaylimits_{\p S}  f\,dy\,dz + g\,dz\,dx + h\,dx\,dy = \mathlarger\iiint\displaylimits_{S} \left(\frac{\p f}{\p x} + \frac{\p g}{\p y} + \frac{\p h}{\p z}\right)dx\,dy\,dz$$
which is the Divergence Theorem.
\end{cproblem}
\begin{solution}
Computing $d\omega$ I get
$$d\omega = \left(\frac{\p f}{\p x} dx + \frac{\p f}{\p y} dy + \frac{\p f}{\p z} dz\right)dy\,dz + \left(\frac{\p g}{\p x} dx + \frac{\p g}{\p y} dy + \frac{\p g}{\p z} dz\right)dz\,dx + \left(\frac{\p h}{\p x} dx + \frac{\p h}{\p y} dy + \frac{\p h}{\p z} dz\right)dx\,dy. $$
Using the properties about differential forms shown in previous problems I simplify to get
$$d\omega =  \frac{\p f}{\p x}dx\,dy\,dz + \frac{\p g}{\p y}dy\,dz\,dx + \frac{\p h}{\p z}dz\,dx\,dy$$
which then simplifies by rules of differential forms to become
$$d\omega = \left( \frac{\p f}{\p x} + \frac{\p g}{\p y} + \frac{\p h}{\p z}\right)dx\,dy\,dz.$$
Given that
$$\boint{\p S}{} \omega = \bigiint{S}{} d\omega $$
By substitution and since $\omega$ is a 2-form defined on a surface with a 2-dimensional boundary $\p S$ and $d\omega$ is a 3-form on $S$, it follows that 
$$\mathlarger\oiint\displaylimits_{\p S}  f\,dy\,dz + g\,dz\,dx + h\,dx\,dy = \mathlarger\iiint\displaylimits_{S} \left(\frac{\p f}{\p x} + \frac{\p g}{\p y} + \frac{\p h}{\p z}\right)dx\,dy\,dz$$
which is the Divergence Theorem.
\end{solution}
\end{document}