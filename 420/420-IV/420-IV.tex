%!TeX program = lualatex
\documentclass[titlepage]{article}
\usepackage{../Head}
\graphicspath{.}
\begin{document}
\fancyhf{}
\fancyhead[RO,R]{Advanced Calculus 420}
\fancyhead[LO,L]{Dakota Wicker}
\fancyhead[CO,C]{Homework IV}
\cfoot{\thepage}

\begin{problem}{1}
Consider the differential one form $\omega=fdx + gdy$. Show that the following are equivalent
\begin{itemize}
\item[a.]$ \underset{L}{\bigointss}\omega = 0 \text{ for any closed loop } L$
\item[b.]$\underset{C}{\bigintss} \omega$ is independent of path $C$ from $X_0$ to $X_1$
\item[c.]$\omega = d\phi $ for some function $\phi(x,y)$
\end{itemize}
\end{problem}
\begin{proof}
To show this equivalency, I will show that a$\implies$b$\implies$c$\implies$a.
First, assume that $\bigointss\displaylimits_{L}\omega = 0$ for any closed loop $L$. Also notice that in $\R^n, \ n\geq2$ there exists at least two paths, $C_1$ and $C_2$, from a point $X_0$ to $X_1$. Which means there is one path, $C_1$, from $X_0$ to $X_1$ and another path, $-C_2$, from $X_1$ to $X_0$. This forms a closed loop $L =C_1-C_2$. So it must be true that,
$$0 = \bigointss\displaylimits_{L}\omega = \bigintss\displaylimits_{C_1}\omega + \bigintss\displaylimits_{-C_2}\omega = \bigintss\displaylimits_{C_1}\omega - \bigintss\displaylimits_{C_2}\omega = 0$$
Which implies that
$$\bigintss\displaylimits_{C_1}\omega = \bigintss\displaylimits_{C_2} \omega$$
This shows that a$\implies$b.\\
\\
Now I show that b$\implies$c. To do this I assume that $\bigintss\displaylimits_{C}\omega$ is independent of path where $C$ is some curve. Now I define a special curve, $C_1$ from $X_0$ to $X_3$. Let $C_2$ be any path from $X_0$ to $X_1$ and let $C_3$ be a path from $X_1$ to $X_3$ such that the $x$ component of $C_3$ does not change. Note that $C_1 = C_2 + C_3$. The work done by a point on this curve, can be described as $\omega = f(x,y)dx + g(x,y)dy$. Now, let $\phi = \bigintsss\displaylimits_{C}\omega$. It follows that 
$$\frac{\p \phi}{\p x} =  \frac{\p}{\p x}\bigintss\displaylimits_{C_1} \omega = \frac{\p}{\p x} \left(\bigintss\displaylimits_{C_2} fdx + gdy + \bigintss\displaylimits_{X_1}^{X_3} fdx + gdy\right)$$ 
and since $C_3$ is not changing in the $x$ direction this implies that 
$$\bigintss\displaylimits_{X_1}^{X_3}fdx + gdy = 0$$
which means that 
$$\frac{\p \phi}{\p x} = \frac{\p}{\p x} \left(\bigintss\displaylimits_{C_2} fdx + gdy + 0\right)  \overset{F.T.C}{\implies} \frac{\p \phi}{\p x}  = f(x,y).$$
Now consider another special, but different curve, $C_4$, from $X_0$ to $X_3$. Let $C_5$ be any path from $X_0$ to $X_2$. Now, let $C_6$ be a path from $X_2$ to $X_3$ such that the $y$ direction does not change. Note that $C_4 = C_5 + C_6$. The work done by a point on this curve can also be described as $\omega$.
Now, I look at $\frac{\p \phi}{\p y}$ from this new curve. That is,
$$\frac{\p \phi}{\p y} = \frac{\p}{\p y}\left(\bigintss\displaylimits_{C_5}fdx + gdy + \bigintss\displaylimits_{X_2}^{X_3} fdx + gdy\right)$$
and since the $y$ component on $C_6$ is constant, this means that
$$\frac{\p \phi}{\p y} = \frac{\p}{\p y}\left(\bigintss\displaylimits_{C_5}fdx + gdy + 0\right) \overset{F.T.C}{\implies} \frac{\p \phi}{\p y} = g$$ 
Finally, I have shown that $\frac{\p \phi}{\p x} = f$ and $\frac{\p \phi}{\p y} = g$.
Since this is true, and by the definition of $d\phi(x,y)$, that, $d\phi = \frac{\p \phi}{\p x} dx + \frac{\p \phi}{\p y}dy$, by substitiution I show that, 
$$d\phi = fdx + gdy = \omega \implies \bigintss\displaylimits_{C} \omega = \bigintss\displaylimits_{C} d\phi.$$
Since $\omega$ is differentiable, $\bigintss\displaylimits_{C} d\phi = \phi$ exists. This shows that b$\implies$c.\\
\\
Finally, to show that c$\implies$a, I assume that $\omega = d\phi$ for some $\phi(x,y)$. Since, 
$$\bigintss\displaylimits_{C}\omega = \bigintss\displaylimits_{C}d\phi = \phi(X_1) - \phi(X_0)$$
where $C$ is a path from $X_0$ to $X_1$. When $C$ is a closed loop, $L$ from $X_0$ to $X_0$, then 
$$\bigointss\displaylimits_{L}\omega = \bigointss\displaylimits_{X_0}^{X_0} d\phi = \phi(X_0) - \phi(X_0) = 0.$$
This shows that c$\implies$a. \\
Since a$\implies$b$\implies$c$\implies$a, this proves that all three statements are equivalent.
\end{proof}

\begin{problem}{2}
Suppose that $\vec{F}$ is a conservative vector field and that $\vec{F} = -\nabla\phi$. Consider a particle of mass $m$ moving along a path $C$ from point $X_0$ to $X_1$. Show that
$$\phi(X_0)+ \frac{1}{2}mv^2\bigg|_{X_0} =  \phi(X_1)+ \frac{1}{2}mv^2\bigg|_{X_1}$$
\end{problem}
\begin{solution}
Using the true statement that $-\bigintss\displaylimits_{C}\vec{F}\bigcdot d\vec{X} = -\bigintss\displaylimits_{C}\vec{F}\bigcdot d\vec{X}$ and that
\begin{align*}
&-\bigintss\displaylimits_{C}\vec{F}\bigcdot d\vec{X} = -\bigintss\displaylimits_{C}m\vec{a}\bigcdot d\vec{X}\\
&= -m\bigintss\displaylimits_{C}\frac{d\vec{v}}{dt} \bigcdot d\vec{X} = \bigintss\displaylimits_{C} d\vec{v}\bigcdot \frac{d\vec{X}}{dt}\\
&= -m\bigintss\displaylimits_{C} \vec{v}\bigcdot d\vec{v} = -m\left(\frac{1}{2}v^2\bigg|_{X_1} - \frac{1}{2}v^2 \bigg|_{X_0}\right)
\end{align*}
and
\begin{align*}
&-\bigintss\displaylimits_{C} \vec{F}\bigcdot d\vec{X} = -\bigintss\displaylimits_{C} \nabla\phi \bigcdot d\vec{X}
\\
&= -\bigintss\displaylimits_{X_0}^{X_1} d\phi = -(\phi(X_1) - \phi(X_0))
\end{align*}
then
\begin{align*}
& -(\phi(X_1) - \phi(X_0)) = -m\left(\frac{1}{2}v^2\bigg|_{X_1} - \frac{1}{2}v^2 \bigg|_{X_0}\right)\\
&\implies \phi(X_0) +  \frac{1}{2}mv^2 \bigg|_{X_0} = \phi(X_1) + \frac{1}{2}mv^2\bigg|_{X_1}
\end{align*}
where 
$\vec{F} = \smat{\frac{\p \phi}{\p x_1}  \\ \vdots \\ \frac{\p \phi}{\p x_n}}.$
\end{solution}

\begin{problem}{3}
In a PTQ, we showed that the electric field $\vec{E}$ located at $P = (x,y,z)$ generated by two point charges, one of charge $-1$ coulombs located at $(0,0,0)$ and the other of charge +1 coulombs located at the point $(1,2,3)$ is given by
$$\vec{E} = \frac{1}{4\pi\epsilon_0}\left( \frac{1}{((x-1)^2 + (y-2)^2 + (z-3)^2)^{\frac{3}{2}}} \bmat{x-1\\y-2\\z-3} - \frac{1}{(x^2 + y^2 + z^2)^\frac{3}{2}} \bmat{x\\y\\z}\right) $$
Show that the work done by $\vec{E}$ to move the particle from point $A$ to point $B$ is independent of path.
\end{problem}
\begin{solution}
Since the work done by the particle moving along the path in the field is
$$\bigintss\displaylimits_{C}\vec{E}\bigcdot d\vec{X}$$
it is clear that since $\vec{E}$ is only a function of $x,y$ and $z$, that the computation of $\vec{E}$ has nothing to do with the path. For example, if there are two curves $C_1$ and $C_2$ that go from $(0,0,0)$ to $(x,y,z)$ then it is true that
$$\bigintss\displaylimits_{C_1} \vec{E} \bigcdot d\vec{X} = \bigintss\displaylimits_{(0,0,0)}^{(x,y,z)} \vec{E} \bigcdot d\vec{X}$$
and 
$$ \bigintss\displaylimits_{C_2} \vec{E} \bigcdot d\vec{X} = \bigintss\displaylimits_{(0,0,0)}^{(x,y,z)} \vec{E} \bigcdot d\vec{X}$$
So, clearly $\bigintss\displaylimits_{C_1} \vec{E} \bigcdot d\vec{X} = \bigintss\displaylimits_{C_2} \vec{E} \bigcdot d\vec{X}$ for any and all paths $C_1$ and $C_2$. Therefore it is path independent.
\end{solution}

\begin{problem}{4}
Suppose $\vec{T}$ is a unit tangent vector to the curve $C$, $\vec{r} = \vec{r}(u)$. Show that the work done in moving a particle in a force field $\vec{F}$ along $C$ is given by $\bigintss\displaylimits_{C}\vec{F}\bigcdot\vec{T}ds$ where $s$ is the arc length.
\end{problem}
\begin{solution}
By using the fact that
$$\vec{T} = \frac{\frac{d\vec{r}}{dt}}{\left|\frac{d\vec{r}}{dt}\right|} = \frac{\frac{d\vec{r}}{dt}}{\frac{ds}{dt}}$$
and the work done on a particle in $\vec{F}$ along $C$ is
$$\bigintss\displaylimits_{C}\vec{F}\bigcdot d\vec{r}$$
So, by substitution I get
$$ \bigintss\displaylimits_{C}\vec{F}\bigcdot\vec{T}ds = \bigintss\displaylimits_{C}\vec{F}\frac{\frac{d\vec{r}}{dt}}{\left|\frac{ds}{dt}\right|}ds = \bigintss\displaylimits_{C}\vec{F}\bigcdot d\vec{r}.$$
This shows that the work done by moving a particle in a force field $\vec{F}$ along $C$ is given by $\bigintss\displaylimits_{C}\vec{F}\bigcdot d\vec{r}$.
\end{solution}
\end{document}