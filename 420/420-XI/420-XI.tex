%!TeX program = lualatex
\documentclass[titlepage]{article}
\usepackage{../Head}
\usepackage{relsize}
\graphicspath{.}
\begin{document}
\fancyhf{}
\fancyhead[RO,R]{Advanced Calculus 420}
\fancyhead[LO,L]{Dakota Wicker}
\fancyhead[CO,C]{Homework XI}
\cfoot{\thepage}

\begin{problem}{1}\ \\
\vspace{-2em}
\begin{itemize}
\item[a.] Prove the following very useful lemma
 \vspace{-1em}\begin{lemma}{I}If $f(x,y,z)$ is continuous on a region $S$ and $\iiint_R f\,dx\,dy\,dz = 0$ for every rectangle $R \subset S$, then $f \equiv 0$ on $S$\end{lemma}
\item[b.]Derive the three-dimensional heat equation governing the temperature $u(x,y,z,t)$ of a three-dimensional uniformly dense solid $S$ at a point $(x,y,z)$ at time $t$.
$$c\rho \frac{\p u}{\p t} = \kappa \left(\frac{\p^2 u}{\p x} + \frac{\p^2 u}{\p y} +\frac{\p^2 u}{\p z}  \right)$$
\end{itemize}
\end{problem}
\begin{solution}
\begin{itemize}
\item[a.] 
\begin{proof}
Suppose that $f(x_0,y_0,z_0) \neq 0 $. W.L.O.G assume $f(x_0,y_0,z_0) > 0$. By continuity there must be a rectangular box $R_0 \subset S$ containing $(x_0,y_0,z_0)$ where $f(x,y,z) \geq \frac{f(x_0,y_0,z_0)}{2}$ 
This implies that
$$\mathlarger\iiint_{R} f\,dx\,dy\,dz \geq \mathlarger\iiint_{R_0} \frac{f(x_0,y_0,z_0)}{2}\,dx\,dy\,dz > 0$$
and when $\iiint_{R_0} f\,dx\,dy\,dz = 0$,
$$\mathlarger\iiint_{R_0} f\,dx\,dy\,dz = 0 \geq \frac{f(x_0,y_0,z_0)}{2}$$
Since this is not possible because $f(x_0,y_0,z_0) > 0$, by contradiction it must be true that if
$\iiint_{R} f\,dx\,dy\,dz = 0$ for every rectangular box $R \subset S$, then $f(x_0,y_0,z_0) \equiv 0$ on $S$.
\end{proof}
\item[b.] 
Since the direction of greatest ascent is the gradient, the greatest descent is going to be the opposite direction so the temperatures velocity vector will have the negative gradient in it. The velocity vector will also grow or shrink by how conductive the material is. So it must be true that the velocity vector of heat will be $\vec{V} = -k\nabla u$.
Let $R$ be an arbitrary rectangular box in $S$. The flux through $R$ is
$$\boint{\p R}{} det\left(\bmat{\vec{V} & \frac{\p \vec{X}}{\p v} & \frac{\p \vec{X}}{\p w}}\right)$$
where $\vec{X}(v,w)$ is the parameterized boundary. This becomes
$$-\kappa\boint{\p R}{} \frac{\p u}{\p x}\,dy\,dz + \frac{\p u}{\p y}\,dz\,dx + \frac{\p u}{\p z}\,dx\,dy =
-\kappa\bigiint{\p R}{} \frac{\p u}{\p x}\,dy\,dz + \frac{\p u}{\p y}\,dz\,dx + \frac{\p u}{\p z}\,dx\,dy$$
and by Divergence Theorem,
$$-\kappa\bigiint{\p R}{} \frac{\p u}{\p x}\,dy\,dz + \frac{\p u}{\p y}\,dz\,dx + \frac{\p u}{\p z}\,dx\,dy =
 -\kappa\mathlarger\iiint_{R} \left(\frac{\p^2 u}{\p x^2} + \frac{\p^2 u}{\p y^2} + \frac{\p^2 u}{\p z^2}\right)\,dx\,dy\,dz.$$
Notice that it is true that the rate at which the heat energy grows is equivalent to the negative of the heat flux on the boundary. 
 In other words, 
 $$\frac{d}{dt}\left(\mathlarger\iiint_{R} c\rho u\,dx\,dy\,dz\right) 
 = \kappa\mathlarger\iiint_{R} \left(\frac{\p^2 u}{\p x^2} + \frac{\p^2 u}{\p y^2} + \frac{\p^2 u}{\p z^2}\right)\,dx\,dy\,dz$$
This can be rewritten as
$$ \mathlarger\iiint_{R} c\rho \frac{d u}{d t}\,dx\,dy\,dz = \kappa\mathlarger\iiint_{R} \left(\frac{\p^2 u}{\p x^2} + \frac{\p^2 u}{\p y^2} + \frac{\p^2 u}{\p z^2}\right)\,dx\,dy\,dz$$
which implies that
$$\mathlarger\iiint_{R} c\rho \frac{d u}{d t} - \kappa\left(\frac{\p^2 u}{\p x^2} + \frac{\p^2 u}{\p y^2} + \frac{\p^2 u}{\p z^2}\right)\,dx\,dy\,dz = 0.$$
Using the lemma previously proven, it must be true that
$$c\rho \frac{d u}{d t} - \kappa\left(\frac{\p^2 u}{\p x^2}+ \frac{\p^2 u}{\p y^2} + \frac{\p^2 u}{\p z^2}\right) = 0.$$
It follows directly that
$$c\rho \frac{d u}{d t} = \kappa\left(\frac{\p^2 u}{\p x^2} + \frac{\p^2 u}{\p y^2} + \frac{\p^2 u}{\p z^2}\right).$$
\end{itemize}
\end{solution}

\begin{problem}{2}
Let a wire around the $z$ axis be defined as $\vec{J} = \smat{0 & 0 & J_3}$
\begin{itemize}
\item[a.] Show that the Biot-Savart Law shows that the magnetic field generated is given by 
$$ \vec{B} = \frac{kJ_3}{x^2 + y^2}\bmat{-y \\ x \\ 0} $$
where $k$ is some positive constant.
\item[b.] Show that if there is a surface $S$ whose boundary does not encircle the $z$-axis, then
$$ \text{work done by } \vec{B} \text{ moving a particle around } \p S = 0$$
\item[c.] Suppose there is a simple closed loop $\Gamma$ encircling the $z$-axis using our normal $x, y$ orientation. Show that
$$ \text{work done by }\vec{B} \text{ moving a particle around } \Gamma = 2\pi kJ_3$$
\item[d.] Use the fact that 
$$\text{work done by }\vec{B} \text{ moving a particle around } \p S = 2\pi k \cdot \text{ flux of } \vec{J} \text{ through S.}$$
to show that
$$\bigiint{S}{} \left(\frac{\p B_3}{\p y} - \frac{\p B_2}{\p z}\right)dy\,dz + \left(\frac{\p B_1}{\p z} - \frac{\p B_3}{\p x}\right)dz\,dx + \left(\frac{\p B_2}{\p x} - \frac{\p B_1}{\p y}\right)dx\,dy $$
is equivalent to
$$2\pi k \bigiint{S}{} J_1\,dy\,dz + J_2\,dz\,dx + J_3\,dx\,dy.$$
\end{itemize}
\end{problem}
\begin{solution}
\begin{itemize}
\vspace{-2em}
\item[a.] Since the magnetic field only goes counterclockwise in a circle around the $z$ axis and the wire is the
$z$ axis, the $z$ component must be zero. Also, since the field must be perpendicular to the wire ($z$ axis),
the $(x,y)$ vector becomes the $(-y,x)$ vector when rotated 90 degrees. So, put all together is
$$ \vec{V} = \bmat{-y \\ x \\ 0}.$$
I also want to divide by the magnitude to get the direction. This is
$$ \frac{1}{\sqrt{(-y)^2 + x^2}} \vec{V}.$$
Now, since $\vec{B}$ is the direction times the magnitude of the current, I will take this direction and
multiply it by the strength of the magnetic field. The strength of the magnetic field is proportional to
the strength of the current. It is also true that the magnetic field is inversely proportional to the distance
from the wire. So, putting this together I get the strength of the current to be  
$$ S = \frac{k J_3}{\sqrt{(-y)^2 + x^2}}$$
Where $k$ is some proportionality constant. So, since $\vec{B}$ is the direction of the field multiplied by the strength of the current, it must be
$$\vec{B} = \frac{k J_3}{\sqrt{(-y)^2 + x^2}} \frac{1}{\sqrt{(-y)^2+(x^2)}}\bmat{-y \\ x \\ 0} = \frac{k J_3}{x^2 + y^2}\bmat{-y \\ x \\ 0}$$
This shows that 
$$ \vec{B} = \frac{k J_3}{x^2 + y^2}\bmat{-y \\ x \\ 0}.$$
\item[b.] The work done by $\vec{B}$ moving a particle around $\p S$ is defined by
$$\boint{\p S}{} \vec{B} \bigcdot d\vec{X} = \boint{\p S}{} B_1\,dx + B_2\,dy + B_3\,dz = \boint{\p S}{} B_1\,dx + B_2\,dy$$ 
Now, applying Stokes' Theorem when $B_3=0$, I get
$$ \boint{\p S}{} \vec{B} \bigcdot d\vec{X} = \bigiint{S}{} \left(\frac{\p B_2}{\p x} - \frac{\p B_1}{\p y}\right)dx\,dy.$$
It follows that taking the partials and simplifying I get
\begin{align*}
\frac{\p B_2}{\p x} &= \frac{\p }{\p x}\left(\frac{xkJ_3}{x^2 + y^2}\right) = \frac{kJ_3(y^2 - x^2)}{(x^2 + y^2)^2} \\
\frac{\p B_1}{\p y} & = \frac{\p }{\p y}\left(\frac{-ykJ_3}{x^2 + y^2}\right) = \frac{kJ_3(y^2 - x^2)}{(x^2 + y^2)^2}
\end{align*}
Plugging this into the formula I obtained for work I get,
$$\bigiint{S}{} \left(\frac{kJ_3(y^2 - x^2)}{(x^2 + y^2)^2}\right) - \left(\frac{kJ_3(y^2 - x^2)}{(x^2 + y^2)^2}\right)dx\,dy = \bigiint{S}{} 0 \,dx\,dy = 0.$$
This shows that the work done by $\vec{B}$ moving a particle around $\p S$ is zero.
\item[c.] It has been shown that the work done by $\vec{B}$ moving a particle around $\p S$ is
$$\boint{\p S}{} B_1\,dx + B_2\,dy$$
and I derive that
\begin{align*}
B_1 &= \frac{-ykJ_3}{x^2 + y^2} \\ 
B_2 &= \frac{xkJ_3}{x^2 + y^2}
\end{align*}
Now, if I perform the transformation to polar coordinates,
$$x = r\cos(\theta), \ y = r\sin(\theta), \ z = 0 $$
where $0\leq \theta\leq 2\pi$, then the work becomes
$$\boint{\p R}{} B_1\,dx + B_2\,dy $$
and 
\begin{align*}
    B_1 &= \frac{-ykJ_3}{x^2 + y^2} \\ 
    B_2 &= \frac{xkJ_3}{x^2 + y^2}
\end{align*}
becomes
\begin{align*}
    B_1 &= \frac{-(r\sin{\theta})kJ_3}{(r\cos(\theta))^2 + (r\sin(\theta))^2} \\ 
    B_2 &= \frac{(r\cos(\theta))kJ_3}{(r\cos(\theta))^2 + (r\sin(\theta))^2}.
\end{align*}

So the work then becomes
$$\boint{\p R}{}\frac{-(r\sin{\theta})kJ_3}{(r\cos(\theta))^2 + (r\sin(\theta))^2}\,dx + \frac{(r\cos(\theta))kJ_3}{(r\cos(\theta))^2 + (r\sin(\theta))^2}\,dy.$$
Since $dx = -r\sin(\theta), \ dy = r\cos(\theta)$, this expands to
$$\boint{\p R}{}\frac{-(r\sin{\theta})kJ_3(-r\sin(\theta))}{(r\cos(\theta))^2 + (r\sin(\theta))^2} + \frac{(r\cos(\theta))kJ_3r\cos(\theta)}{(r\cos(\theta))^2 + (r\sin(\theta))^2} d\theta.$$
which then simplifies to become 
$$\boint{\p R}{}\frac{-\sin{\theta}kJ_3(-r\sin(\theta))}{r} + \frac{(r\cos(\theta))kJ_3r\cos(\theta)}{r} d\theta $$
which becomes 
$$kJ_3\boint{0}{2\pi} d\theta = 2\pi kJ_3.$$
This shows that the work done by $\vec{B}$ moving a particle around $\Gamma$ is $2\pi kJ_3$.
\item[d.] It has been shown that the work done by $\vec{B}$ moving a particle around $\p S$ is 
$$\boint{\p S}{} \vec{F} \bigcdot d\vec{X} = \boint{\p S}{} B_1\,dx + B_2\,dy + B_3\,dz.$$
Applying Stokes' Theorem to this I get that the work done is equivalent to,
$$\bigiint{S}{} \left(\frac{\p B_3}{\p y} - \frac{\p B_2}{\p z}\right)dy\,dz + \left(\frac{\p B_1}{\p z} - \frac{\p B_3}{\p x}\right)dz\,dx + \left(\frac{\p B_2}{\p x} - \frac{\p B_1}{\p y}\right)dx\,dy.$$
It is also true that the flux through $S$ is 
$$\bigiint{S}{} \vec{J}\bigcdot d\vec{S}$$
and when $S$ is parameterized, this is equivalent to 
$$\bigiint{R}{} \text{det}\left(\bmat{\vec{J} & \frac{\p S}{\p u} & \frac{\p S}{\p v}}\right)du\,dv
= \bigiint{S}{} J_1\,dy\,dz + J_2\,dz\,dx + J_3\,dx\,dy.$$
Since it is a fact that the work done by $\vec{B}$ moving a particle around  $\p S$ is $2\pi k \cdot$ flux of  $\vec{J}$ through S, then by substiution it must be true that
$$ \bigiint{S}{} \left(\frac{\p B_3}{\p y} - \frac{\p B_2}{\p z}\right)dy\,dz + \left(\frac{\p B_1}{\p z} - \frac{\p B_3}{\p x}\right)dz\,dx + \left(\frac{\p B_2}{\p x} - \frac{\p B_1}{\p y}\right)dx\,dy $$ 
$$ = $$
$$2\pi k \bigiint{S}{} J_1\,dy\,dz + J_2\,dz\,dx + J_3\,dx\,dy.$$
\end{itemize}
\end{solution}
\end{document}