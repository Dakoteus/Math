%!TeX program = lualatex
\documentclass[titlepage]{article}
\usepackage{../Head}
\graphicspath{.}
\begin{document}
\fancyhf{}
\fancyhead[RO,R]{Advanced Calculus 420}
\fancyhead[LO,L]{Dakota Wicker}
\fancyhead[CO,C]{Homework XI}
\cfoot{\thepage}

\begin{problem}{1}
Consider the formula for the work done moving a mass m along the path $\vec{X}(t)$ from $X_0$ to $X$ which is
$$ \bigintsss_{t=t_0}^{t_1} \vec{F}(t) \bigcdot d\vec{X}$$
Show that
$$ \bigintsss_{t=t_0}^{t_1} \vec{F}(t) \bigcdot d\vec{X} = \frac{1}{2}mv^2(t_1) - \frac{1}{2}mv^2(t_0)$$
\end{problem}

\begin{solution}
First I make a substitution
$$ \bigintsss_{t=t_0}^{t_1} m\vec{a} \bigcdot d\vec{X}$$
Then I rewrite this as
$$ m\bigintsss_{t=t_0}^{t_1} \frac{d\vec{v}}{dt} \bigcdot d\vec{X} = m\bigintsss_{t=t_0}^{t_1} d\vec{v} \bigcdot \frac{d\vec{X}}{dt} = m\bigintsss_{t=t_0}^{t_1} \vec{v}\bigcdot d\vec{v}.$$
Note that $\vec{v}\bigcdot d\vec{v} = v_xdv_x + v_ydv_y$ by definition of dot product. Now, by substitution I get
$$m\bigintsss_{t=t_0}^{t_1} \vec{v}\bigcdot d\vec{v} = m\bigintsss_{t=t_0}^{t_1} v_xdv_x + v_ydv_y$$
which means that
\begin{align*}
m\bigintsss_{t=t_0}^{t_1} \vec{v}\bigcdot d\vec{v} &= m\left(\bigintsss_{t=t_0}^{t_1} v_xdv_x + \bigintsss_{t=t_0}^{t_1} v_ydv_y\right) \\
&= m \left(\frac{1}{2}v_x^2(t_1) - \frac{1}{2}v_x^2(t_0)\right) + \left(\frac{1}{2}v_y^2(t_1) - \frac{1}{2}v_y^2(t_0)\right) \\
&= \frac{1}{2}m\left(\left(v_x^2(t_1) -v_x^2(t_0)\right) + \left(v_y^2(t_1) - v_y^2(t_0)\right)\right) \\
&= \frac{1}{2}m\left(v_x^2(t_1) -v_x^2(t_0) + v_y^2(t_1) - v_y^2(t_0)\right) \\
&= \frac{1}{2}m\left(v_x^2(t_1) -v_y^2(t_1) + v_x^2(t_0) - v_y^2(t_0)\right) \\ 
&= \frac{1}{2}m(-v^2(t_1) + -v^2(t_0)) = \frac{1}{2}m(v^2(t_1) - v^2(t_0)) \\ 
&= \frac{1}{2}m v^2(t_1) - \frac{1}{2}mv^2(t_0))
\end{align*}
\end{solution}

\begin{problem}{2a}
Show that the work done by gravity moving the mass along the path $\vec{X}(t)$ from $X_0$ to $X$ is also given by 
$$\bigintss_{t=t_0}^{t_1} \vec{F}(t) \bigcdot d\vec{X} = mg(y(t_1)-y(t_0))$$ 
\end{problem}

\begin{solution}
First I make a substitution using the fact that $\vec{F} = m\vec{a}$. That is,
 $$\bigintss_{t=t_0}^{t_1} \vec{F} \bigcdot d\vec{X} = \bigintss_{t=t_0}^{t_1} m\vec{a} \bigcdot d\vec{X}.$$
Then using the fact that $\vec{a} = \begin{bmatrix} 0 \\ g \end{bmatrix}$, I make the substitution,
$$  \bigintss_{t=t_0}^{t_1} m\vec{a} \bigcdot d\vec{X} =\bigintss_{t=t_0}^{t_1} m \begin{bmatrix} 0 \\ g \end{bmatrix} \bigcdot d\vec{X}.$$
Now by rewriting what I have I get 
\begin{align*}
&\bigintss_{t=t_0}^{t_1} m \begin{bmatrix} 0 \\ g \end{bmatrix} \bigcdot d\vec{X}\\
= m &\bigintss_{t=t_0}^{t_1}  \begin{bmatrix} 0 \\ g \end{bmatrix} \bigcdot \begin{bmatrix} dx(t) \\ dy(t) \end{bmatrix} \\
= m &\bigintss_{t=t_0}^{t_1} gdy(t) \\
= mg&\bigintss_{t=t_0}^{t_1} dy(t) = mg(y(t_1) - y(t_0)) 
\end{align*}
This shows that
$$\bigintss_{t=t_0}^{t_1} \vec{F}(t) \bigcdot d\vec{X} = mg(y(t_1)-y(t_0))$$
is the work done by gravity.
\end{solution}

\begin{problem}{2b}
Use our two formulas for work to show that if the mass starts from rest then
$$v(t) = \sqrt{2g(y(t_1) - y(t_0))} $$
\end{problem}
\begin{solution}
Since the initial state is at rest, this means velocity is equal to zero at $t_0$. That is, $v(t_0) = 0$. Plugging this into the equation I got in the first problem I get,
$$\bigintsss_{t=t_0}^{t_1} \vec{F}(t) \bigcdot d\vec{X} = \frac{1}{2}mv^2(t_1) - 0$$
Plugging this into the equation I got in the second problem I get,
$$\bigintss_{t=t_0}^{t_1} \vec{F}(t) \bigcdot d\vec{X} = mg(y(t_1)-y(t_0)). $$
This means that
$$\bigintss_{t=t_0}^{t_1} \vec{F}(t) \bigcdot d\vec{X} = \frac{1}{2}mv^2(t_1) = mg(y(t_1)-y(t_0)).$$
Using algebraic manipulation I get
\begin{align*}
 &\frac{1}{2}mv^2(t_1) = mg(y(t_1)-y(t_0)) \\
 &\implies \frac{1}{2}v^2(t_1) = g(y(t_1)-y(t_0)) \\ 
 &\implies v^2(t_1) =2g(y(t_1) - y(t_0)) \\
 &\implies v(t) = \sqrt{2g(y(t_1) - y(t_0))}
 \end{align*}
 This shows that $v(t) = \sqrt{2g(y(t_1) - y(t_0))}$ when the mass is initially at rest.
\end{solution}

\begin{problem}{3a}
Use the above information to show that the time it takes for the mass to travel from $X_0$ to $X_1$ under the influence of gravity (starting from rest) is given by
$$\bigintss_{X=X_0}^{X_1}dt = \bigintss_{X=X_0}^{X_1}\frac{ds}{\sqrt{2g(y(t_1)-y(t_0))}}$$
\end{problem}
\begin{solution}
It is a fact that $\frac{ds}{dt} = v(t)$. Using this fact it is clear that $dt =\frac{ds}{v(t)}$. Using the equation I got from problem 2a (because we are accounting for gravity), I can rewrite $v(t) = \sqrt{2g(y(t_1)-y(t_0))}$. So, rewriting the given integral I get,
$$\bigintss_{X=X_0}^{X_1}dt = \bigintss_{X=X_0}^{X_1} \frac{ds}{v(t)} = \bigintss_{X=X_0}^{X_1} \frac{ds}{\sqrt{2g(y(t_1)-y(t_0))}}$$
This shows the time it takes for the mass to travel from $X_0$ to $X_1$ under the influence of gravity.
\end{solution}

\begin{problem}{3b}
Use the formula from 3a to show that the time it takes for the mass to move along the cycloid
\begin{align*}
&x=a(\theta - \sin(\theta)) \\
&y=a(1-\cos(\theta)) \\ 
&\theta_0\leq \theta \leq \pi
\end{align*}
is $\pi\sqrt{\frac{a}{g}}$ which is independent of $\theta_0.$
\end{problem}

\begin{solution}
When I use the formula
$$ \bigintss_{X=X_0}^{X_1} \frac{ds}{v(t)} = \bigintss_{X=X_0}^{X_1} \frac{ds}{\sqrt{2g(y(t_1)-y(t_0))}} $$
I can solve for ds knowing that it is the change in arc length. That is,
$$ds = \sqrt{\left(\frac{dx}{dt}\right)^2 + \left(\frac{dy}{dt}\right)^2} = \sqrt{a^2(\cos(\theta)-1)^2 + (a\sin(\theta))^2}.$$
When I plug this into the integral I get 
$$ \bigintss_{X=X_0}^{X_1} \frac{\sqrt{a^2(\cos(\theta)-1)^2 + (a\sin(\theta))^2}}{\sqrt{2g(y(t_1)-y(t_0))}}  $$

(idk notation is very weird I will ask questions)
\end{solution}
\end{document}